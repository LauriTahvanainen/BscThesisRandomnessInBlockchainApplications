\chapter{Määritelmät\label{methods}}

\section{Hajautusfunktio}
Hajautusfunktio h on funktio joka 

\section{Lohkoketju}

Lohkoketju on hajautettu, lupavapaa ja muuttumaton tilikirja. Perinteisesti tilikirjalla tarkoitetaan tilien kokoelmaa, johon on merkitty kaikki tilien olemassaolon aikana tapahtuneet transaktiot. Uutta tietoa on mahdollista lisätä vain tilikirjan loppuun lisäämällä uusia transaktioita. Lohkoketjun tapauksessa tilikirja tarkoittaa tietokantaa, johon tallennetaan tilien transaktioita ja uutta tietoa voi lisätä tietokantaan vain luomalla uusia transaktioita, jotka lisätään tietokantaan lohkossa. Tiivistäen lohko koostuu transaktioista, sekä edellisen lohkon hajautusarvosta. Edellisen lohkon hajautusarvo liittää uuteen lohkoon aikeisemmat transaktiot, muodostaen lohkoketjun. Tilejä lohkoketjussa hallinnoidaan julkisen avaimen salauksella. Salaisella avaimella allekirjoitetaan uusia transaktioita, jolla hallinnoidaan tilinä toimivaa julkisesta avaimesta johdettua osoitetta. Uusi transaktio julkaistaan kaikille osallistujille odottamaan, että tietokannan päivittämisestä vastaavat konsensusmekanismia suorittavat osallistujat lisäävät transaktion lohkoketjuun.

Lohkoketju on lupavapaa, mikä tarkoittaa sitä, että kuka tahansa voi osallistua tietokannan päivittämiseen lähettämällä uusia transaktioita, sekä lisäämällä transaktioita lohkoissa tietokantaan suorittamalla ketjun konsensusalgoritmia. Osallistujan tunnisteena toimii vain tämän salausavaimet, mikä mahdollistaa nimettömän osallistumisen. Lupavapauden johdosta järjestelmään voi osallistua myös haitallisia toimijoita, jotka haluavat esimerkiksi lähettää itseään hyödyntäviä virheellisiä transaktioita. Järjestelmän ollessa hajautettu, jotkut osallistujat voivat myös kohdata teknisiä ongelmia ja lopettaa osallistumisen, tai jopa alkaa lähettämään virheellisiä transaktioita haluamattaan. Osallistujilla pitää olla mekanismi jolla saavutetaan konsensus siitä mikä on tietokannan nykyinen validi tila kuvaillussa ympäristössä. Lohkoketjun osallistujat suorittavat konsensusalgoritmia tietokannan nykyisen validin tilan selvittämiseksi. 

Yleistäen on olemassa kaksi suosittua konsensusmekanismia. Työtodistus(Proof of Work, PoW)-konsensusmekanismia, käyttää kirjoitushetkellä esimerkiksi Bitcoin, sekä Ethereum lohkoketjut. PoW-konsensusta kutsutaan myös Nakamoto-konsensukseksi, jonka tarkempi kuvaus löytyy Bitcoin-whitepaperista \cite{Nakamoto_bitcoin}. Uudempi ja yhä yleistyvä mekanismi on  osakkuustodistus(Proof of Stake, PoS). 

\subsection{PoW}
PoW-konsensuksessa uusien transaktioiden lohko liitetään lohkoketjuun pääpiirteittäin seuraavan protokollan mukaisesti: Jokainen konsensukseen osallistuja, joita kutsutaan kaivajiksi (Miner) kilpailee uuden lohkon lisäämisestä. Kaivaja valitsee uusista transaktioista joukon uutta lohkoa varten ja tarkistaa transaktioiden oikeellisuuden. Transaktiot, edellisen lohkon hajautusarvo, aikaleima sekä osallistujan valitsema satunnainen arvo (Nonce) syötetään hajautusfunktiolle, Bitcoinin tapauksessa SHA256. Aikaisempien lohkojen perusteella on sovittu tämänhetkistä vaikeutta kuvaava luku $v$. Jos saadun hajautusarvon alussa on peräkkäin vähintään $v$ 0-merkkiä on lohko hyväksyttävä. Tällöin kaivaja julkaisee uuden lohkon verkolle. Lohkon kanssa lähetetään metadataa, kuten esimerkiksi löydetty nonce, edellisen lohkon hajautusarvo sekä aikaleima. Muut kaivajat voivat vielä vahvistaa lohkon oikeellisuuden suorittamalla yllä mainitun transaktioiden validoinnin ja hajautusarvon laskemisen uudestaan ja varmistamalla, että saatu hajautusarvo on löytäjän ilmoittama. Kun uusi lohko on validoitu, aloittavat kaivajat prosessin alusta edellisenä lohkona toimien nyt löydetty validoitu lohko. Kaivajat yrittävät aina lisätä uutta lohkoa pisimmän validin ketjun jatkeeksi. Löytäjä saa palkkion uuden lohkon luomisesta.

Jos hajautusarvo ei alakaan halutulla määrällä nollia, kaivaja vaihtaa nonce-arvoa ja laskee hajautusarvon uudestaan. Sopivan hajautusarvon löytämiseen ei ole hajautusfunktion alkukuva resistenssin (Preimage resistance) myötä muuta keinoa kuin yrittää satunnaisia nonce-arvoja. Arvoa $v$ muuttamalla voidaan vaikuttaa siihen kuinka kauan sopivan nonce-arvon löytämisessä kestää keskimäärin tietyllä laskentateholla. Nonce-arvon esittäminen onkin todiste, että uuden lohkon lisäämisen eteen on tehty työtä. Protokolla toimii niin kauan, kuin lohkojen luomiseen kohdistuvasta laskentatehosta yli $50 \%$ on rehellisten kaivajien hallussa. Lohkoketjussa aikaisemmin olevien lohkojen transaktioiden muuttaminen vaatisi, että laskentatyö tehdään uudestaan muutetun sekä kaikkien sitä seuraavien lohkojen kohdalla.

Tutkielman kannalta kiinnostavaa PoW-konsensusmekanismissa on se, että nonce-arvo, kuten siitä johdettu lohkon hajautusarvo, ei ole kenenkään ennustettavissa. PoW-lohkoketjujen lohkojen hajautusarvoa onkin mahdollista käyttää satunnaisuuden lähteenä. Hajautusarvon käyttöä ja siihen liittyviä ongelmia käsitellään luvussa kolme.

\subsection{PoS}

PoS konsensusmekanismissa lohkoja tuottaa validaattoreiksi kutsuttu joukko osallistujia. Uuden lohkon tuottaja valitaan validaattoreiden joukosta satunnaisesti ja mekanismi vaatiikin protokollan hajautettuun satunnaisuuden generointiin toimiakseen luotettavasti. Paino, joka määrää validaattorin todennäköisyyden tulla valituksi lohkon tuottajaksi määräytyy validaattorin tilin saldon myötä. Osallistuja joka ei ole validaattori voi delegoida oman saldonsa validaattorille palkintoa vastaan, jolloin validaattorin todennäköisyys tulla valituksi nousee suhteessa muihin validaattoreihin. Mekanismin idea on, että sillä lohkoketjun valuutalla on arvoa, on mekanismissa haitallisten osallistujien omattava mittava määrä taloudellista arvoa, jotta he voivat manipuloida luotavia lohkoja. Onnistunut hyökkäys myös alentaisi valuutan arvoa, haitaten hyökkääjää, joka todennäköisesti hyökkää nimenomaan valuutan arvon takia. Käytössä on usein bysanttilaisia ongelmia sietävä konsensusmekanismi ja siten haitallisilla osallistujilla voi olla maksimissaan $1/3$ osallistujista.

\section{Lohkoketjusovellus}

Lohkoketjusovelluksella tarkoitetaan sovellusta, jonka ydinlogiikan suoritus tapahtuu lohkoketjun solmuissa, eli sovellusta suoritetaan lohkoketjun protokollan mukaisessa hajautetussa verkossa. Sovelluksen tila tallennetaan lohkoketjun hajautettuun tietokantaan. Esimerkiksi älysopimuksista koostuva, Ethereum\cite{ethereum_foundation_ethereum_nodate}
 lohkoketjussa suoritettava, sovellus on esimerkki lohkoketjusovelluksesta. Yksi lohkoketjusovelluksen ominaispiirre on lupavapaus, joka tarkoittaa, että kuka tahansa voi kommunikoida sovelluksen kanssa. Sovelluksia tukevan lohkoketjualustan tavoitteena on usein myös, että sovelluksia ei voida sensuroida, tarkoittaen, että sovellusten suorittamista ei voi estää mikään yksittäinen toimija.
Lohkoketjun lupavapaa luonne vaatii satunnaislukujen generoimisen protokollan olevan hajautettu, manipuloimaton, ennustamaton ja julkisesti todennettavissa.

Tässä tapauksessa satunnaislukujen generoimisen ongelma rajataan tarkemmin seuraavasti:
Yhden satunnaisluvun generoimiseen osallistuu $N$ osallistujaa,  $N > 2$. Lohkoketjuympäristössä jokaisella osallistujalla on lohkoketjussa käytettävän asymmetrisen salausmenetelmän tuottamat julkinen avain P ja yksityinen avain S, joiden avulla on vahvistettavissa, että tietty viesti on tietyn osallistujan lähettämä. Kuka tahansa voi liittyä generointiprotokollaan ennen protokollan suorittamisen aloittamista, mutta uusia osallistujia ei voi liittyä kesken protokollan suorittamisen. Tilanne vastaa monen sovelluksen käyttötapausta, kuten esimerkiksi lottoarvonnan. Kun arvonta alkaa, ei uusia osallistuja voi enää liittyä. Generoitu satunnaisluku julkaistaan lohkoketjuun. 

\section{Satunnaisuus}

Protokollan generoimiin satunnaislukuihin kohdistetaan seuraavat vaatimukset \cite{simic_review_2020}:
\begin{itemize}
    \item[--] 1: Protokollan generoimat satunnaisluvut noudattavat satunnaisjakaumaa
    \item[--] 2: Protokollan generoimia satunnaislukuja ei voi ennustaa
    \item[--] 3: On julkisesti todennettavissa, että protokollan generoimat satunnaisluvut on generoitu protokollaa noudattaen.
    \item[--] 4: Generoitava satunnaisluku ei ole kenenkään manipuloitavissa
\end{itemize}\textbf{}
Käyttötarkoituksesta riippuen voidaan vaatia, että:

\begin{itemize}
    \item[--] 5: Jos protokollaa suorittaa vähintään 1 rehellinen osallistuja, ei protokollaa voi estää generoimasta satunnaislukua.
    \item[--] 6: Protokollan suorittamiseen kuluu aikaa enintään t.
    \item[--] 7: Yksittäisen osallistujan kommunikointikompleksisuus on enintään x, esim. O(n)
\end{itemize}

\section{Vahvistettava Satunnaisfunktio (Verifiable Random Function, VRF)}

\section{Vahvistettava Viivefunktio (Verifiable Delay Function, VDF)}