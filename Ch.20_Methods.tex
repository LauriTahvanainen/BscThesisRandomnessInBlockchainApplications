\chapter{Määritelmät\label{methods}}

\section{Kryptografiset primitiivit}

Seuraavat kryptografiset primitiivit ovat oleellisia lohkoketjun rakennuspaloja ja niitä käytetään lohkoketjusovelluksissa.

\textit{Tiivistefunktio} on funktio h, joka liittää mielivaltaisen pituiset syötteet saman pituisiin maalijoukon alkioihin, tiivisteisiin. Binäärimuodossa: 
\begin{equation}
    H : \{0, 1\}^* \longrightarrow  \{0, 1\}^l \text{, missä $l$ on tiivisteen pituus}
\end{equation}

Jotta tiivistefunktio olisi \textit{kryptografinen tiivistefunktio}, sen tulee täyttää kolme ominaisuutta \cite{cryptoeprint:2011:565}: 
\begin{enumerate}
    \item Funktio on vahvasti törmäyskestävä (Collision Resistant), eli on laskennallisesti kannattamatonta löytää kaksi syötettä, jotka tuottavat saman tiivisteen.
    \item Funktio on alkukuvakestävä (Pre-image resistance), eli on laskennallisesti kannattamatonta löytää tiivistettä vastaava syöte.
    \item Funktio on heikosti törmäyskestävä, eli jos annettuna on syöte, on laskennallisesti kannattamatonta löytää toinen syöte, jonka tiiviste on sama.
\end{enumerate}

Tutkielmassa tiivistefunktioiden oletetaan olevan kryptografisia, ja käytetään vain termiä tiivistefunktio. Esimerkiksi Ethereum-lohkoketjussa on käytössä Keccak-256 tiivistefunktio. 

Lohkoketjussa käytetään \textit{Julkisen avaimen salausta} transaktioiden allekirjoittamiseen. Asymmetrisessä avainparissa on salainen avain (SK), sekä salaisesta avaimesta johdettu julkinen avain (PK). Avainpari toimii lohkoketjussa osallistujan ainoana tunnisteena, mikä mahdollistaa avoimen ja anonyymin osallistumisen. Esimerkiksi Ethereum-ketjussa on käytössä elliptisiin käyriin nojaava ECDSA-algoritmi (Elliptic Curve Digital Signature Algorithm). 

%\section{Verkon aika}
%Tutkielmassa oletetaan, että käsiteltyjä protokollia suoritetaan internetin välityksellä asynkronisessa verkossa. Osallistujat eivät välttämättä jaa yhteistä kelloa ja lähetettyjen viestien välillä voi olla vaihteleva viive. Lohkoketjussa osallistujat jakavat lohkoketjun muodostaman ajan. Lohkoketju pyrkii tuottamaan uusia lohkoja keskimäärin vakioajassa. Järjestelmän aikayksikkönä voidaan käyttää lohkon numeroa. Esimerkiksi Ethereum muodostaa uuden lohkon noin 15 sekunnin välein. Yhteisen ajan myötä protokollaan on mahdollista määritellä aikakatkaisu, jonka kuka tahansa voi laukaista.

\section{Lohkoketju}

Lohkoketju on hajautettu ja muuttumaton tilikirja \cite{8760539}. Perinteisesti tilikirjalla tarkoitetaan tilien kokoelmaa, johon on merkitty kaikki tilien olemassaolon aikana tapahtuneet transaktiot. Uutta tietoa on mahdollista lisätä vain tilikirjan loppuun lisäämällä uusia transaktioita.

Lohkoketjussa uusia transaktioita voi lisätä vain luomalla lohkoketjun jatkeeksi uuden lohkon. Yksinkertaistaen, lohko koostuu transaktioista, sekä edellisen lohkon tiivisteestä. Edellisen lohkon tiiviste liittää uuteen lohkoon aikaisemmat lohkot, muodostaen lohkoketjun. Lohkoketjun käyttäjät lähettävät tiliään koskevia transaktioita allekirjoittamalla transaktion salaisella avaimella, josta tilin osoite on johdettu. Uusi transaktio julkaistaan odottamaan, että ketjun päivittämisestä vastaavat konsensusmekanismia suorittavat käyttäjät lisäävät transaktion lohkoketjuun. 

Lohkoketjussa on siis kaksi joukkoa käyttäjiä. Käyttäjien joukko, jotka lähettävät uusia transaktioita, sekä joukko, jotka suorittavat konsensusmekanismia lisätäkseen lohkoketjuun uusia transaktioita lohkoissa. Lohkoketju voi olla lupavapaa (permissionless) tai luvallinen (permissioned). Lupavapaassa lohkoketjussa kuka tahansa yksittäinen internet-yhteyden omaava käyttäjä voi kuulua vapaasti kumpaan tahansa mainittuun joukkoon. Tämän lisäksi kenen tahansa on mahdollista tarkastella kaikkia lohkoketjussa aikaisemmin tapahtuneita transaktioita. Luvallisessa lohkoketjussa transaktioiden lähettämistä, konsensusmekanismiin osallistumista tai transaktioiden tarkastelua on rajattu tietylle joukolle käyttäjistä. Tutkielmassa oletetaan, että lohkoketjusovellusta suoritetaan lohkoketjussa, jossa kenellä tahansa on vähintään oikeus uusien transaktioiden lähettämiseen, sekä aikaisempien transaktioiden tarkastelemiseen. Konsensusmekanismiin osallistumisessa voi olla rajoitteita, mutta osallistuminen ei ole täysin suljettu ennaltamäärätylle joukolle.

Konsensusmekanismi on protokolla jolla lohkoketjun käyttäjät saavuttavat yhteisymmärryksen siitä mikä on ketjun nykyinen validi tila. Käyttäjät suorittavat konsensusalgoritmia ketjun nykyisen validin tilan selvittämiseksi ja uusien transaktioiden lisäämiseksi lohkoissa. Tutkielmassa huomioidaan kaksi merkittävintä konsensusmekanismia. Työtodistus (Proof of Work, PoW)-konsensusmekanismia, käyttää kirjoitushetkellä esimerkiksi Bitcoin \cite{Nakamoto_bitcoin}, sekä Ethereum \cite{buterin_ethereum_2014} lohkoketjut. PoW-konsensusta kutsutaan myös Nakamoto-konsensukseksi, jonka tarkempi kuvaus löytyy Bitcoin-valkopaperista \cite{Nakamoto_bitcoin}. Uudempi ja yhä yleistyvä mekanismi on osakkuustodistus(Proof of Stake, PoS), jonka esimerkkinä toimii esimerkiksi Algorand-lohkoketjun konsensusmekanismi \cite{gilad_algorand_2017}. 

%\subsection{Työtodistus}
PoW-konsensuksessa jokainen konsensukseen osallistuja, joita kutsutaan kaivajiksi (Miner) kilpailee uuden lohkon lisäämisestä. Esitetään yksinkertaistetusti Bitcoin-valkopaperissa \cite{Nakamoto_bitcoin} esitellyn työtodistuksen toiminta.

Kaivaja valitsee uusista transaktioista joukon uutta lohkoa varten ja tarkistaa transaktioiden oikeellisuuden. Transaktiot, edellisen lohkon tiiviste, aikaleima sekä osallistujan valitsema satunnainen arvo (Nonce) syötetään tiivistefunktiolle. Aikaisempien lohkojen perusteella on sovittu vaikeutta kuvaava luku $v$. Jos saadun tiivisteen numeerinen arvo on pienempi kuin $v$ on lohko hyväksyttävä. Tällöin kaivaja julkaisee uuden lohkon verkolle. Muut kaivajat voivat vielä vahvistaa lohkon oikeellisuuden suorittamalla yllä mainitun transaktioiden validoinnin ja tiivisteen laskemisen uudestaan löytäjän parametreilla ja varmistamalla, että saatu tiiviste vastaa löytäjän ilmoittamaa. Kun uusi lohko on validoitu, aloittavat kaivajat prosessin alusta syötteenään uusi transaktioiden joukko, validoidun lohkon tiiviste, aikaleima, sekä nonce. Kaivajat yrittävät aina lisätä uutta lohkoa pisimmän validin ketjun jatkeeksi. 

Jos tiiviste ei alakaan halutulla määrällä nollia, kaivaja vaihtaa nonce-arvoa ja laskee tiivisteen uudestaan. Sopivan tiivisteen löytämiseen ei ole tiivisteen alkukuvakestävyyden myötä muuta keinoa kuin yrittää satunnaisia nonce-arvoja. Arvoa $v$ muuttamalla voidaan vaikuttaa siihen kuinka kauan sopivan nonce-arvon löytämisessä kestää keskimäärin tietyllä laskentateholla ja arvo vaihtelee niin, että uusi lohko löydettäisiin keskimäärin aina samassa ajassa. Lohkoketjussa aikaisemmin olevien lohkojen transaktioiden muuttaminen vaatisi, että laskentatyö tehdään uudestaan muutetun sekä kaikkien sitä seuraavien lohkojen kohdalla. Konsensusmekanismi olettaa, että lohkojen luomiseen kohdistuvasta laskentatehosta yli $50\%$ on rehellisten kaivajien hallussa, jolloin lohkoketjun muuttumattomuus on taattu, sillä rehelliset kaivajat kaivavat aina pisintä ketjua.

Löytäjä saa palkkiona uuden lohkon luomisesta vastaluotua valuuttaa, kuten myös transaktiokuluja. Transaktiokulu on transaktion lähettäjän määräämä arvo, joka lisätään lähetettävän arvon päälle, joten kaivajat valitsevat lohkoon transaktioita joissa on suurimmat transaktiokulut.

Tutkielman kannalta kiinnostavaa PoW-konsensusmekanismissa on se, että nonce-arvo, kuten myös siitä johdettu lohkon tiiviste, ei ole kenenkään ennustettavissa. PoW-lohkoketjujen lohkojen tiivistettä onkin mahdollista käyttää satunnaisuuden lähteenä. Tiivisteen käyttöä satunnaisuuden lähteenä ja siihen liittyviä ongelmia käsitellään luvussa \ref{results}.


% Kuinka validaattorien joukko muodostetaan?
Osakkuustodistusta hyödyntävässä konsensusmekanismissa lohkon tuottaja valitaan käyttäjien tilien saldojen perusteella \cite{NguyenCongT2019PCMf}. Lohkon tuottajaksi valittu käyttäjä tuottaa lohkon ja muut käyttäjät äänestävät lohkon oikeellisuudesta. Asynkronisessa verkossa konsensuksen taattu saavuttaminen deterministisellä konsensusprotokollalla on mahdotonta, jos edes yksi osallistuja toimii virheellisesti \cite{fischer_impossibility_1985}. Tämän takia osakkuustodistuksessa valinta tehdään usein satunnaisesti, jolloin käytettävä konsensusprotokolla on epädeterministinen. Paino, joka määrää käyttäjän todennäköisyyden tulla valituksi lohkon tuottajaksi määräytyy käyttäjän tilin saldon mukaan. Koska lohkoketjun valuutalla on arvoa, on mekanismissa hyökkääjien omattava mittava määrä taloudellista arvoa, jotta he voivat luoda haluamiaan lohkoja. Usein oletus on, että $2/3$ ketjun arvosta on rehellisien käyttäjien tileillä. 

 PoS-ketjujen konsensusalgoritmit tarvitsevat hajautetusti tuotettua satunnaisuutta lohkon luojan arpomiseen. Joitain tutkielmassa esiteltäviä protokollia voi siten käyttää myös konsensusmekanismin osana. Satunnaisuutta hyödyntävästä PoS-konsensusmekanismista toimii esimerkkinä Algorand-lohkoketjun konsensusmekanismi \cite{gilad_algorand_2017}. PoS-ketjun lohkon tiivistettä ei voi pitää PoW-ketjun lohkon tiivisteen tapaan satunnaisena, sillä lohkon luo yksi validaattori, joka voi vaikuttaa tiivisteeseen valitsemalla lohkoon transaktioita, jotka tuottavat suotuisan tiivisteen.

\section{Lohkoketjusovellus}

Lohkoketjun voi nähdä myös tilakoneena. Jokainen transaktio edistää tilakonetta edellisten lohkojen määrittämän tilan pohjalta. Lohkoketjuun voidaan pelkkien tilien saldojen sijaan tallentaa kokonaisia ohjelmia. Näin lohkoketju toimii Turing-täydellisenä laskenta-alustana ja lohkoketju muodostaa virtuaalikoneen, jolla kuka tahansa voi suorittaa mitä tahansa muuttumatonta laskentaa lupavapaasti.

Tämän idean toteutti ensimmäisenä lohkoketju Ethereum \cite{buterin_ethereum_2014}. Ethereum-lohkoketjussa ohjelmia suoritetaan lohkoketjussa pinopohjaisella Ethereum virtuaalikoneella (Ethereum Virtual Machine, EVM). EVM suorittaa tavukoodia, mutta ohjelmia voi kirjoittaa korkean tason ohjelmointikielellä, esimerkiksi Solidity:llä\footnote{https://docs.soliditylang.org}. Lohkoketjuun tallennettuja ja suoritettavia ohjelmia kutsutaan älysopimuksiksi (Smart Contract) \cite{buterin_ethereum_2014}.  Älysopimuksella on osoite ja rajapintoja joita osallistujat kutsuvat transaktiossa edistääkseen älysopimuksen laskentaa, sekä pysyvää muistia, mikä on käytettävissä eri transaktioiden välillä. Transaktion lähettäjä maksaa laskennan edistämisestä transaktiokuluna lohkoketjun virtuaalivaluuttaa laskennan määrän mukaan. Yhden transaktion aikana suoritettavalle laskennan määrälle on myös yläraja. Kirjoittamishetkellä Ethereum-ketjussa sopimuksena määritellyn valuutan, ERC-20 tokenin, siirtäminen osoitteesta toiseen vaati kuluina noin 5.28\$ edestä Ethereum-valuuttaa \cite{etherscanio_ethereum_nodate}. Erityisen kallista on sopimuksen pysyvään muistiin tallentaminen.

\textit{Lohkoketjusovelluksella} tarkoitetaan sovellusta, jonka ydinlogiikan suoritus tapahtuu lohkoketjussa. Käyttäjät voivat suorittaa ohjelman vaatimaa laskentaa lohkoketjun älysopimuksissa (On-chain) lähettämällä transaktoita, tai lohkoketjun ulkopuolella (Off-chain) omalla laitteellaan. Off-chain laskenta on käyttäjälle transaktiokulujen takia moninkertaisesti halvempaa kuin on-chain laskenta. Sovelluksen käyttäjien välinen viestintä tapahtuu lohkoketjun kautta, osallistujat eivät siis lähetä suoraan toisilleen viestejä. Sovellus on lohkoketjun tapaan lupavapaa.

Satunnaisuutta tuottavaan protokollaan, joka suorittaa laskentaa lohkoketjussa, kohdistuu vaatimus mahdollisimman pienistä kommunikointikustannuksista On-chain laskennan vaatimien transaktiokulujen myötä. Älysopimuksia tukevissa lohkoketjuissa on viime vuosina tapahtunut merkittävää kehitystä ja jo nyt toiset lohkoketjut mahdollistavat Ethereum-ketjua moninkertaisesti pienemmät kulut. Esimerkiksi kirjoitushetkellä Algorand-lohkoketjun transaktiokulu oli 0.00091\$ arvosta Algorand-virtuaalivaluuttaa \cite{noauthor_algorand_nodate}. Älysopimuslaskennan suuren taloudellisen kustannuksen ongelma ei ole kuitenkaan täysin poistunut erityisesti sovelluksissa, joissa käyttäjien on lähetettävä tuhansia transaktioita tai suoritettava pitkäkestoista laskentaa.

\section{Lohkoketjusovelluksen käyttäjät}

Satunnaislukuja tarvitsevaan lohkoketjusovellukseen oletetaan osallistuvan $N$ käyttäjää, $N > 1$, joista jokaisella on käytössään lohkoketjussa käytetty julkisen avaimen salauksen avainpari, jonka mahdollistamilla allekirjoituksilla käyttäjä osoittaa identiteettinsä. Julkista avainta voidaan kutsua tutkielmassa lohkoketjuympäristössä myös nimellä \textit{tili} tai \textit{osoite}. Täten esimerkiksi sillä, että käyttäjä lähettää transaktion osoitteestaan tarkoitetaan, että käyttäjä lähettää transaktion allekirjoittaen sen yksityisellä avaimellaan, jota vastaa muille näkyvä julkinen osoite.  Satunnaisuutta tuottavien protokollien kannalta on huomioitavaa, että yksittäinen käyttäjä voi generoida itselleen haluamansa määrän osoitteita ja näyttäytyä täten muille osallistujille monina erillisinä käyttäjinä.

Lohkoketjusovelluksen lupavapauden myötä sovellusta voi käyttää myös haitalliset käyttäjät, jotka haluavat esimerkiksi lähettää itseään hyödyntäviä transaktioita. Haitallisia käyttäjiä kutsutaan myös hyökkääjiksi. Hyökkääjän laskentatehon oletetaan olevan korkeintaan polynomisesti suurempi kuin rehellisten osallistujien. Hyökkääjä ei siis kykene murtamaan vahvoiksi oletettuja kryptograafisia työkaluja, kuten esimerkiksi julkisen avaimen salausta. Hyökkääjä ei myöskään kykene estämään, viivyttämään tai väärentämään toisten osallistujien lähettämiä transaktioita. Haitallisiin käyttäjiin lasketaan myös osallistujat, jotka kohtaavat teknisiä ongelmia, jotka estävät käyttäjää lähettämästä transaktioita. Rehellinen osallistuja lähettää aina protokollan vaatimia transaktioita.

Tutkielmassa oletetaan, että jokaisella osallistujalla on käytössään satunnaisuuden lähde, jota toiset osallistujat eivät voi ennustaa tai manipuloida. Rehellinen osallistuja käyttää aina tätä lähdettä, kun tältä kysytään protokollassa satunnaista syötettä.

\section{Satunnaisuuden vaatimukset}

Älysopimus voi saada satunnaisuutensa kolmannelta osapuolelta tai sovellus voi toteuttaa itse protokollan jolla tuottaa satunnaisuutta. Esimerkiksi lottoarvonnan osallistujat voivat tuottaa itse voittajan määräävän satunnaisluvun. Protokolla voidaan toteuttaa myös yksittäisenä sovelluksena, jonka satunnaisuutta muut sovellukset käyttävät. Tällöin protokolla toimii satunnaisuuden majakkana (Randomness Beacon).

Lohkoketjusovellusta varten satunnaislukuja tuottaviin menetelmiin kohdistetaan seuraavat vaatimukset, jotka menetelmien on vähintään täytettävä, jotta niiden tuottamaa satunnaisuutta voidaan käyttää lohkoketjusovelluksessa turvallisesti. Vaatimukset perustuvat aikaisemman tutkimuksen esittämiin vaatimuksiin \cite{cherniaeva2019homomorphic, schindler_hydrand_2020, syta_scalable_2017}. 

\begin{enumerate}
    \item Yksittäisen osallistujan ei ole mahdollista estää satunnaisluvun tuottamista.
    \item Tuotettavia satunnaislukuja ei voi ennustaa.
    \item On julkisesti todennettavissa, että menetelmän tuottamat satunnaisluvut on tuotettu menetelmää noudattaen.
    \item Tuotettava satunnaisluku ei ole kenenkään manipuloitavissa.
\end{enumerate}

Vaatimusten lisäksi seuraavilla menetelmien ominaisuuksilla on merkitystä menetelmän käytettävyyden ja käytännön toteutuksen kannalta:

\begin{enumerate}
    \item Mikä määrä $t$ sovelluksen käyttäjistä $t \leq N$ voi olla haitallisia, niin että käytetty menetelmä kuitenkin tuottaa vaatimusten 1, 2, 3 ja 4 mukaisen luvun?
    \item Mikä on yksittäisen käyttäjän kommunikointikompleksisuus? Esim $O(N)$.
    \item Kestääkö menetelmä Sybil hyökkäyksiä?
\end{enumerate}

Kommunikointikompleksisuudella tarkoitetaan tutkielman tapauksessa sitä kuinka monta transaktiota yksittäinen osallistuja joutuu lähettämään osallistuakseen yhden satunnaisluvun tuottamiseen lohkoketjusovellukselle. 

Sybil hyökkäyksellä \cite{douceur2002sybil} tarkoitetaan hyökkäystä, jossa yksi käyttäjä esiintyy järjestelmässä monella identiteetillä. Yhden käyttäjän luomat identiteetit näyttäytyvät muille käyttäjille erillisinä käyttäjinä. Lohkoketjusovellus on lupavapaa ja käyttäjän identifioi vain osoite, joita yksittäinen käyttäjä voi generoida haluamansa määrän. Jos lohkoketjusovellukseen, joka toteuttaa esimerkiksi hajautetun satunnaisuutta tuottavan protokollan, voi liittyä vain transaktiokulun hinnalla, voi sovellus olla altis Sybil hyökkäykselle. Jos sovellus ei ole altis Sybil hyökkäykselle sanotaan, että sovellus on Sybil resistentti.