\chapter{Yhteenveto\label{conclusions}}

Tutkielmassa esiteltiin ja vertailtiin menetelmiä, joilla lohkoketjusovellus voi saada käyttöönsä satunnaisuutta. Satunnaisuutta tuottavan menetelmän satunnaisuuden pitää olla julkisesti todennettavaa, manipuloimatonta ja ennustamatonta. Satunnaisuuden hyödyntäminen ei ole triviaalia julkisessa, deterministisessä ja lupavapaassa lohkoketjuympäristössä. Menetelmien monimutkaisuus vaihtelee aina yksinkertaisista tiivistefunktioita käyttävistä, monimutkaisempia hajautettuja kryptografisia metodeja käyttäviin menetelmiin. Metodien käytöllä pyritään pakottamaan käyttäjät toimimaan sovitulla tavalla, tai vähentämään hyökkääjien mahdollisuuksia. Sovellus voi saada satunnaisuuden ulkopuolelta tai sovelluksen käyttäjät voivat tuottaa itse satunnaisuutta. Tutkielman pääpaino oli sovelluksen käyttäjien itse tuottamassa satunnaisuudessa. Tutkielma toimii myös yleiskatsauksena hajautetun satunnaisuuden tuottamisen ongelmaan.

Eri menetelmien tekemät oletukset, ominaisuudet ja heikkoudet vaihtelevat, minkä myötä menetelmien soveltuvuus eri sovelluksiin vaihtelee. Vaihtelevuuden myötä tutkielmassa ei pyritty löytämään yhtä ongelmaa ratkaisevaa menetelmää vaan keskityttiin vertailemaan menetelmien ominaisuuksia ja heikkouksia niin, että vertailusta olisi apua uutta satunnaisuutta vaativaa lohkoketjusovellusta kehittäessä. Menetelmien analysoinnissa käytettiin mittareina yhdelle käyttäjälle koituvia kommunikointikustannuksia sekä kuinka monta epärehellistä käyttäjää menetelmä kestää. Analysoinnissa nostettiin myös huomioita menetelmien käytännön toteutuksiin liittyen.

Tutkielmassa käytiin läpi hyökkäyksiä, joita menetelmiä kohtaan voidaan kohdistaa. Osasta satunnaisuutta tuottavista menetelmistä huomattiin, että ne tarvitsevat tavan menetelmän Sybil resistenssin parantamiseksi jotta protokollan voi toteuttaa turvallisesti. Sybil resistenssin parantaminen lohkoketjusovelluksessa nostettiin yhdeksi jatkotutkimuksen aiheeksi.

%% Lähdeluettelossa otsikko yhteenveto