\chapter{Yhteenveto\label{conclusions}}

Tutkielmassa esiteltiin ja vertailtiin menetelmiä joilla lohkoketjusovellus voi saada käyttöönsä satunnaisuutta. Satunnaisuuden hyödyntäminen ei ole triviaalia deterministisessä ja lupavapaassa lohkoketjuympäristössä. Metodien monimutkaisuus vaihtelee aina yksinkertaisista tiivistefunktioita käyttävistä monimutkaisempia kryptografisia metodeja käyttäviin menetelmiin. Sovellus voi saada satunnaisuuden ulkopuolelta tai sovelluksen käyttäjät voivat tuottaa itse satunnaisuutta. Tutkielman pääpaino oli sovelluksen käyttäjien itse tuottamassa satunnaisuudessa. Tutkielma toimii myös yleiskatsauksena hajautetun satunnaisuuden tuottamisen ongelmaan.

Eri menetelmien tekemät oletukset, ominaisuudet ja heikkoudet vaihtelevat, minkä myötä menetelmien soveltuvuus eri sovelluksiin myös vaihtelee. Vaihtelevuuden myötä tutkielmassa ei pyritty löytämään yhtä ongelmaa ratkaisevaa menetelmää vaan keskityttiin vertailemaan menetelmien ominaisuuksia ja heikkouksia niin, että vertailusta olisi apua uutta satunnaisuutta vaativaa lohkoketjusovellusta kehittäessä. Menetelmien analysoinnissa käytettiin mittareina yhdelle käyttäjälle koituvia kommunikointikustannuksia sekä kuinka monta epärehellistä käyttäjää menetelmä kestää. Analysoinnissa nostettiin myös huomioita menetelmien käytännön toteutuksiin liittyen.

Satunnaisuutta tuottavan menetelmän tarjoaman satunnaisuuden pitää olla julkisesti todennettavaa, manipuloimatonta ja ennustamatonta. Tutkielmassa käytiin läpi hyökkäyksiä, joita menetelmiä kohtaan voidaan kohdistaa. Osasta satunnaisuutta tuottavista menetelmistä huomattiin, että ne tarvitsevat tavan menetelmän sybil resistenssin parantamiseksi jotta protokollan voi toteuttaa turvallisesti. Sybil resistenssin parantaminen lohkoketjusovelluksessa nostettiin yhdeksi jatkotutkimuksen aiheeksi.