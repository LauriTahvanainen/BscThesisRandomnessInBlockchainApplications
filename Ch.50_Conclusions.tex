\chapter{Yhteenveto\label{conclusions}}

Julkisen, todennettavan ja manipuloimattoman satunnaisuuden generoiminen lohkoketjusovellusten tarpeisiin on haastavaa johtuen lohkoketjun lupavapaudesta. Lohkoketjusovellukseen soveltuvan protokollan on pidettävä osallistujan kommunikointikustannus mahdollisimman pienenä, jossa auttaa lohkoketjun oman entropian sekä osallistujien syöttämän entropian käyttäminen protokollassa. Käsitellyt protokollat pyrkivät viivyttämään generoitavan satunnaisluvun julkistamista niin, että hyökkääjä ei ehdi simuloimaan protokollan tulosta ennen kuin protokollasta irtautuminen on myöhäistä.

Muita protokollia satunnaisuuden generoimiseen lohkoketjusovelluksien tarkoituksiin joita tutkielmassa ei käsitelty, on esimerkiksi Merlin-ketjuja hyödyntävä protokolla \cite{MerlinChains}, julkista todennettavaa salaisuuksien jakamista käyttävät kommitointiprotokollat \cite{syta_scalable_2017}, homomorfista salausta käyttävät protokollat kuten myös peliteoriaa ja taloudellisia kannustimia hyödyntävät protokollat.