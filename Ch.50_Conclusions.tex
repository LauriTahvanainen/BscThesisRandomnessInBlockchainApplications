\chapter{Yhteenveto\label{conclusions}}

Tutkielmassa esiteltiin ja vertailtiin menetelmiä joilla lohkoketjusovellus voi saada käyttöönsä satunnaisuutta. Satunnaisuuden hyödyntäminen ei ole triviaalia deterministisessä ja lupavapaassa lohkoketjuympäristössä. Metodien monimutkaisuus vaihtelee aina yksinkertaisista tiivistefunktioita käyttävistä monimutkaisempia kryptografisia metodeja käyttäviin menetelmiin. Sovellus voi saada satunnaisuuden ulkopuolelta tai sovelluksen käyttäjät voivat tuottaa itse satunnaisuutta. Tutkielman pääpaino oli sovelluksen käyttäjien itse tuottamassa satunnaisuudessa. Eri menetelmien tekemät oletukset, ominaisuudet ja heikkoudet vaihtelevat, minkä myötä menetelmien soveltuvuus eri sovelluksiin myös vaihtelee. Vaihtelevuuden myötä tutkielmassa ei pyritty löytämään yhtä ongelmaa ratkaisevaa menetelmää vaan keskityttiin vertailemaan menetelmien ominaisuuksia ja heikkouksia niin, että vertailusta olisi apua uutta satunnaisuutta vaativaa lohkoketjusovellusta kehittäessä. Tutkielma toimii myös yleiskatsauksena hajautetun satunnaisuuden tuottamisen ongelmaan.

Osasta satunnaisuutta tuottavista protokollista huomattiin, että ne tarvitsevat tavan jolla parantaa protokollan Sybil resistenssiä jotta protokollan voi toteuttaa turvallisesti. Sybil resistenssin parantaminen lohkoketjusovelluksessa nostettiin yhdeksi jatkotutkimuksen aiheeksi.

Hajautetun satunnaisuuden tuottamiseen tarkoitettuja protokollia vertaillaan viimeaikaisissa tutkimuksissa \cite{bhat2021randpiper, bhat2022optrand, schindler_hydrand_2020}. Tutkielmassa ei käyty läpi mainituissa tutkimuksissa esiintyviä menetelmiä, jotka käyttävät BLS \cite{boneh2001short} allekirjoituksia. Tutkielmassa ei esitelty myöskään peliteoreettisia menetelmiä satunnaisuuden tuottamiseen.