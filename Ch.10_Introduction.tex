\chapter{Johdanto\label{intro}}

Satunnaisuutta on jo pitkään hyödynnetty osana hajautettujen järjestelmien toimintaa parantamaan järjestelmien luotettavuutta sekä turvallisuutta. M. Rabin esitti vuonna 1983 satunnaisuutta generoivan majakan (Randomness Beacon) käsitteen, jonka avulla voidaan toteuttaa esimerkiksi varmennettujen sähköpostien lähettämisen, sekä sopimusten allekirjoittamisen protokollat \cite{rabin_transaction_1983}. Yksi keskitetty satunnaisuutta generoivan majakan toteutus on NIST-majakka \cite{computer_security_division_interoperable_2019}. Majakka toimii luotettuna kolmantena osapuolena, mutta hajautettuun järjestelmään voi kohdistua vaatimus siitä, että se kykenee generoimaan itse tarvitsemansa satunnaisuuden jolloin satunnaisuuden generoimiseen tarvitaan hajautettu protokolla. Kolmannen osapuolen palveluna käytettävä majakka voi myös itsessään olla toteutettu hajautettua protokollaa käyttäen esimerkiksi vikasietoisuuden parantamiseksi. Nykypäivänä luotettavalle satunnaisuuden generoimiselle on tarvetta esimerkiksi julkisen salauksen avaimien generoimisessa \cite{corrigan-gibbs_ensuring_2014}, joka on operaatio, jonka oikeelliseen toimintaan nojaavat internetin käytetyimmät salatun viestinnän protokollat. 

Viime vuosina yleistyneet lohkoketjuteknologiat ovat esimerkki hajautetuista järjestelmistä joilla on tarve luotettavalle satunnaisuudelle. Jotkin lohkoketjuprotokollat, kuten myös lohkoketjusovellukset tarvitsevat satunnaisuutta toimiakseen. Lohkoketjuprotokollan tasolla satunnaisuutta hyödynnetään esimerkiksi uusissa konsensusalgoritmeissa \cite{gilad_algorand_2017, hanke_dfinity_2018}, jotka kuluttavat perinteisissä lohkoketjuissa, kuten Bitcoin \cite{noauthor_bitcoin_nodate} ja Ethereum \cite{noauthor_ethereum_nodate}, käytettävää Nakamoto-konsensusalgoritmia vähemmän energiaa konsensuksen saavuttamiseen. Satunnaisuutta käyttävät lohkoketjut pystyvät käsittelemään myös selvästi perinteisiä lohkoketjuja suuremman määrän transaktioita sekunnissa.

Lohkoketjun deterministinen luonne luo haasteen lohkoketjuympäristössä suoritettaville lohkoketjusovelluksille jotka vaativat oikein toimiakseen satunnaisuutta. Tälläisia sovelluksia ovat esimerkiksi hajautetut lotto-arvonnat \cite{pooltogether_pooltogether_nodate}, jotka tarvitsevat satunnaislukuja arvonnan suorittamiseen, sekä hajautetut tuomioistuimet \cite{lesaege_kleros_2020}, jotka tarvitsevat satunnaisuutta tuomareiden valitsemiseen. Lohkoketjuympäristö on järjestelmänä lupavapaa tarkoittaen, että kuka tahansa voi liittyä järjestelmään ja vuorovaikuttaa sen kanssa. Tämä luo järjestelmään laitteistovirheiden lisäksi uhan haitallisista toimijoista, mikä vaikeuttaa satunnaisuutta generoivien protokollien suunnittelua, sillä haitallisilla toimijalla on intressi ennustaa tai manipuloida käytettyä satunnaisuutta. Lohkoketjuteknologia mahdollistaa taloudellisen arvon vaivattoman siirtämisen järjestelmän käyttäjien välillä ja haitallisia toimijoita voi lohkoketjusovelluksissa motivoida satunnaisuuden ennustamiseen tai manipulointiin kymmenien, jopa satojen, miljoonien arvoiset palkkiot. Toisaalta arvon siirtämisen mahdollisuuden myötä lohkoketjuympäristössä satunnaisuuden generoimisen protokolliin voidaan lisätä taloudellisia ja peliteoreettisia kannustimia haitallisten toimijoiden torjumiseksi.

Tässä tutkielmassa kokoan yhteen olemassaolevia protokollia hajautettuun satunnaislukujen generoimiseen.
Käsittelen protokollia erityisesti lohkoketjusovellusten näkökulmasta ja vaikka osa protokollista on suunnitelty nimenomaan lohkoketjuympäristöön, niin esittelen tutkielmassa myös protokollia jotka eivät vaadi toimiakseen lohkoketjuympäristöä ja jotka voidaan täten toteuttaa perinteisessä hajautetussa ympäristössä. Luvussa kaksi määrittelen tarvittavat käsitteet, minkä jälkeen luku kolme esittelee kootut protokollat sekä analysoi miten protokollat täyttävät vaatimukset satunnaisuudesta, ennustamattomuudesta, todennettavuudesta sekä manipuloimattomuudesta. Luvussa analysoidaan myös protokollien kommunikointivaatimuksia ja esitetään hyökkäyksiä joita protokollia vastaan voidaan kohdistaa. Luvussa neljä vertaillaan eri protokollia ja tarkastellaan protokollien sopivuutta lohkoketjusovelluksiin. Luku viisi toimii yhteenvetona.