\chapter{Johdanto\label{intro}}

Lohkoketjusovelluksilla on kymmeniä tuhansia päivittäisiä käyttäjiä \cite{noauthor_state_nodate}. Sovellukset pyrkivät tarjoamaan erilaisia ratkaisuja aina terveystietojen hallinnasta\footnote{https://medibloc.com/en/} hajautettuihin rahoitusmarkkinoihin\footnote{https://ethereum.org/en/defi/}. Lohkoketjusovelluksen ydinlogiikan suoritus tapahtuu lohkoketjussa. Lohkoketjuteknologia mahdollistaa sellaisten sovellusten toteuttamisen, joissa sovelluksen käyttäminen on avointa ja käyttäjät voivat käyttää sovelluksia anonyymisti eikä käyttäjien tarvitse silti luottaa toisiinsa eikä mihinkään kolmanteen osapuoleen.

Lohkoketjun deterministisyys, avoimuus ja käyttäjien anonyymiteetti luo kuitenkin haasteen lohkoketjusovelluksille, jotka vaativat oikein toimiakseen satunnaisuutta. Tälläisia sovelluksia ovat esimerkiksi hajautetut lotto-arvonnat\footnote{https://pooltogether.com}, jotka tarvitsevat satunnaislukuja arvonnan suorittamiseen, sekä hajautetut tuomioistuimet, jotka tarvitsevat satunnaisuutta tuomareiden valitsemiseen \cite{lesaege_kleros_2020}. 

Lohkoketjusovelluksia uhkaa haitalliset toimijat, mikä vaikeuttaa satunnaisuuden tuottamista lohkoketjusovellusta varten, sillä haitallisilla toimijoilla on intressi ennustaa tai manipuloida lohkoketjusovelluksessa käytettyä satunnaisuutta. Lohkoketjuteknologia mahdollistaa taloudellisen arvon vaivattoman siirtämisen järjestelmän käyttäjien välillä ja haitallisia toimijoita voi lohkoketjusovelluksissa motivoida satunnaisuuden ennustamiseen tai manipulointiin kymmenien, jopa satojen, miljoonien arvoiset palkkiot. Toisaalta arvon siirtämisen mahdollisuuden myötä lohkoketjuympäristössä satunnaisuutta tuottaviin menetelmiin voidaan lisätä taloudellisia ja peliteoreettisia kannustimia haitallisten toimijoiden torjumiseksi.

Satunnaisuuden tuottamista hajautetussa ympäristössä, jossa käyttäjät eivät luota toisiinta on tutkittu jo 1980-luvulla \cite{10.1145/1008908.1008911}. Vaikka suurin osa tutkielmassa esiteltävistä menetelmistä on suunniteltu nimenomaan lohkoketjuympäristöön, niin esittelen tutkielmassa myös satunnaisuuden tuottamisen protokollia jotka eivät vaadi toimiakseen lohkoketjua, mutta ne voidaan toteuttaa myös lohkoketjujen kontekstissa. Protokollat pyrkivät vastaamaan kysymykseen: Kuinka käyttäjät, jotka eivät luota toisiinsa, voivat tuottaa satunnaisuutta? Hajautettua satunnaisuutta tuottavia protokollia hyödynnetään lohkoketjusovellusten lisäksi osana lohkoketjun toimintaa. Esimerkiksi tietyissä konsensusalgoritmeissa \cite{gilad_algorand_2017, hanke_dfinity_2018}, joiden avulla lohkoketjun osallistujat saavuttavat yhteisymmärryksen lohkoketjun nykyisestä tilasta, tarvitaan hajautetusti tuotettua satunnaisuutta.

% Kuinka tuottaa tai käyttää?
Tässä tutkielmassa kokoan yhteen olemassaolevia menetelmiä, jotka tarjoavat ratkaisun kysymykseen: Kuinka lohkoketjusovellus saa hyödynnettäväkseen satunnaisuutta? Luvussa \ref{methods} määrittelen tarvittavat käsitteet ja vaatimukset satunnaisuudelle. Luku \ref{results} esittelee menetelmiä, joilla lohkoketjusovellukseen saadaan satunnaisuutta sekä analysoi miten hyvin nämä menetelmät täyttävät satunnaisuudelle asetetut vaatimukset. Tutkielman pääpaino on protokollissa, joiden avulla sovelluksen käyttäjät tuottavat itse sovelluksen vaatiman satunnaisuuden. Luvussa käydään läpi eri menetelmien oletuksia, kommunikointikustannuksia, esitetään hyökkäyksiä, joita menetelmiä vastaan voidaan kohdistaa sekä esitetään huomioita menetelmien toteuttamisesta lohkoketjusovelluksessa. Luvussa \ref{discussion} vertaillaan eri menetelmiä ja tarkastellaan niiden sopivuutta lohkoketjusovelluksiin. Luku \ref{conclusions} toimii yhteenvetona.