\chapter{Johdanto\label{intro}}

Satunnaisuutta on jo pitkään hyödynnetty osana hajautettujen järjestelmien toimintaa parantamaan järjestelmien luotettavuutta sekä turvallisuutta. M. Rabin esitti vuonna 1983 kuinka satunnaisuutta voidaan hyödyntää varmennettujen sähköpostien lähettämisen sekä sopimusten allekirjoittamisen mahdollistavissa protokollissa \cite{rabin_transaction_1983}. Nykypäivänä luotettavalle satunnaisuuden tuottamiselle on tarvetta esimerkiksi julkisen salauksen avaimien generoimisessa \cite{corrigan-gibbs_ensuring_2014}, joka on operaatio, jonka oikeelliseen toimintaan nojaavat internetin käytetyimmät salatun viestinnän protokollat, kuten esimerkiksi TLS \cite{rescorla_transport_2018}. 

Lohkoketjuteknologiat ovat esimerkki hajautetuista järjestelmistä joissa tarvitaan myös satunnaisuutta. Lohkoketjuprotokollat, kuten myös jotkin lohkoketjusovellukset vaativat satunnaisuutta toimiakseen. Lohkoketjuprotokollan tasolla satunnaisuutta hyödynnetään esimerkiksi tietyissä konsensusalgoritmeissa \cite{gilad_algorand_2017, hanke_dfinity_2018}, joiden avulla lohkoketjun osallistujat saavuttavat yhteisymmärryksen lohkoketjun nykyisestä tilasta. 

Lohkoketjuympäristössä suoritetaan lohkoketjusovelluksia. Lohkoketju on järjestelmänä lupavapaa tarkoittaen, että kuka tahansa voi liittyä järjestelmään ja vuorovaikuttaa sen kanssa. Lohkoketjuteknologia mahdollistaa sellaisten sovellusten toteuttamisen, joissa käyttäjien ei tarvitse luottaa toisiinsa eikä mihinkään kolmanteen osapuoleen.

Lohkoketjun deterministisyys luo kuitenkin haasteen lohkoketjusovelluksille, jotka vaativat oikein toimiakseen satunnaisuutta. Tälläisia sovelluksia ovat esimerkiksi hajautetut lotto-arvonnat \cite{pooltogether_pooltogether_nodate}, jotka tarvitsevat satunnaislukuja arvonnan suorittamiseen, sekä hajautetut tuomioistuimet \cite{lesaege_kleros_2020}, jotka tarvitsevat satunnaisuutta tuomareiden valitsemiseen. 

Lupavapaus luo järjestelmään uhan haitallisista toimijoista, mikä vaikeuttaa satunnaisuutta tuottavien protokollien suunnittelua, sillä haitallisilla toimijoilla on intressi ennustaa tai manipuloida lohkoketjusovelluksessa käytettyä satunnaisuutta. Lohkoketjuteknologia mahdollistaa taloudellisen arvon vaivattoman siirtämisen järjestelmän käyttäjien välillä ja haitallisia toimijoita voi lohkoketjusovelluksissa motivoida satunnaisuuden ennustamiseen tai manipulointiin kymmenien, jopa satojen, miljoonien arvoiset palkkiot. Toisaalta arvon siirtämisen mahdollisuuden myötä lohkoketjuympäristössä satunnaisuuden tuottaviin protokolliin voidaan lisätä taloudellisia ja peliteoreettisia kannustimia haitallisten toimijoiden torjumiseksi.

% Kuinka tuottaa tai käyttää?
Tässä tutkielmassa kokoan yhteen olemassaolevia metodeja, jotka tarjoavat ratkaisun kysymykseen: Kuinka lohkoketjusovellus saa hyödynnettäväkseen satunnaisuutta? Luvussa kaksi määrittelen tarvittavat käsitteet ja vaatimukset satunnaisuudelle. Luku kolme esittelee metodeja, joilla lohkoketjusovellukseen saadaan satunnaisuutta sekä analysoi miten hyvin nämä metodit täyttävät asetetut vaatimukset. Tutkielman pääpaino on kommunikointiprotokollissa, joiden avulla sovelluksen käyttäjät tuottavat itse sovelluksen vaatiman satunnaisuuden. Luvussa analysoidaan eri metodien kommunikointivaatimuksia, esitetään hyökkäyksiä, joita metodeja vastaan voidaan kohdistaa sekä esitetään huomioita metodien toteuttamisesta lohkoketjusovelluksessa. Luvussa neljä vertaillaan eri metodeja ja tarkastellaan metodien sopivuutta lohkoketjusovelluksiin. Luku viisi toimii yhteenvetona.

Vaikka suurin osa metodeista on suunniteltu nimenomaan lohkoketjuympäristöön, niin esittelen tutkielmassa myös satunnaisuuden tuottamisen protokollia jotka eivät vaadi toimiakseen lohkoketjuympäristöä, mutta ne voidaan toteuttaa myös lohkoketjuympäristössä. Protokollat pyrkivät vastaamaan myös yleisempään kysymykseen: Kuinka käyttäjät, jotka eivät luota toisiinsa, voivat tuottaa satunnaisuutta? 