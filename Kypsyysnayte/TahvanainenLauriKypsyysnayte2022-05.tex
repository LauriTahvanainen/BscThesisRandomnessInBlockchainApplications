\documentclass{article}

\title{ Miksi satunnaisuutta tarvitaan lohkoketjuissa, ja miten se voidaan
toteuttaa? \\\small{Kypsyysnäyte}}
\author{Lauri Tahvanainen}
\usepackage[finnish]{babel}
\begin{document}
\maketitle

Lohkoketju on hajautettu ja muuttumaton tilikirja \cite{8760539}. Lohkoketju koostuu lohkoista, jotka koostuvat yksinkertaistaen tilejä koskevista transaktioista, sekä edellisen lohkon kryptografisesta tiivisteestä \cite{cryptoeprint:2011:565}. Viite edelliseen lohkoon muodostaa lohkoketjun. Ketjun käyttäjät suorittavat konsensusmekanismia, jotta kaikki käyttäjät pääsevät yhteisymmärrykseen siitä, mikä on ketjun nykyinen validi tila. Konsensusmekanismia suorittaessa lisätään uusia transaktioita lohkoissa. 

Jos kuka tahansa internet-yhteyden omaava käyttäjä voi luoda uusia transaktioita, osallistua konsensusmekanismin suoritukseen, sekä tarkastella lohkoketjun kaikkia aikaisempia transaktioita, niin sanotaan, että lohkoketju on lupavapaa. Bitcoin \cite{Nakamoto_bitcoin} on ensimmäinen lupavapaa lohkoketju. Bitcoinin Nakamoto-konsensuksessa uutta lohkoa lisäävä käyttäjä (kaivaja) yrittää löytää ennen muita kaivajia sellaisen lohkon, jonka tiiviste täyttää edellisten lohkojen määrittämän vaatimuksen. Tällaisen lohkon löytäminen vaatii kaivajalta paljon laskentaa ja Nakamoto-konsensusta kutsutaankin työtodistukseksi (Proof of Work). Uuden lohkon löytäjä saa palkkioksi lohkoketjun virtuaalivaluuttaa. Mekanismi olettaa, että enemmistö käyttäjistä yrittää lisätä valideja transaktioita.

Bitcoin konsensusmekanismin suorittamisen sähkönkulutus on suurempi kuin monen valtion\footnote{https://digiconomist.net/bitcoin-energy-consumption}. Bitcoin ei myöskään pysty käsittelemään suurta määrää transaktioita sekunnissa, eikä se siten skaalaudu suurelle käyttäjämäärälle. Pienempää energiankulutusta sekä skaalautuvuutta varten on kehitetty lohkoketjuja, jotka hyödyntävät perinteisiä bysanttilaisia virheitä kestäviä konsensusmekanismeja lohkoketjun tilan määrittämiseksi. Tiedetään, että asynkronisessa verkossa konsensuksen saavuttaminen deterministisellä konsensusprotokollalla on mahdotonta, jos edes yksi osallistuja toimii virheellisesti \cite{fischer_impossibility_1985}. Lohkoketjun käyttäjät toimivat asynkronisessa internetissä ja kyseisen tuloksen myötä lohkoketju vaatii epädeterministisen konsensusmekanismin vikasietoisuuden saavuttamiseksi. Lohkoketju tarvitsee siis satunnaisuutta toimiakseen. 

Yleisin skaalautuvampi ja energiatehokkaampi konsensusmekanismi on osakkuustodistus (Proof of Stake). Algorand \cite{gilad_algorand_2017} lohkoketju toimii hyvänä esimerkkinä osakkuustodistuksesta. Osakkuustodistuksessa yksi käyttäjä valitaan satunnaisesti luomaan uusi lohko. Muut käyttäjät vahvistavat lohkon äänestämällä. Todennäköisyys lohkon luojan valintaan määräytyy sen mukaan, paljon käyttäjällä on varallisuutta tilillään lohkoketjussa. Mekanismi olettaa, että enintään kolmannes varallisuudesta on haitallisten käyttäjien hallussa.

Lohkoketju voidaan nähdä tilikirjan sijaan myös tilakoneena, jolla voi suorittaa mitä tahansa Turing täydellistä laskentaa. Esimerkkinä tästä toimii Ethereum-lohkoketju, jossa tileihin voi tallentaa älysopimuksiksi kutsuttuja ohjelmia \cite{buterin_ethereum_2014}. Lohkoketjussa älysopimuksena suoritettava ohjelma, kuten esimerkiksi hajautettu lottoarvonta\footnote{https://pooltogether.com/} voi myös tarvita satunnaisuutta.  
Koska lohkoketju, sekä lohkoketjusovellukset ovat hajautettuja, lupavapaita ja julkisia vaaditaan satunnaisuuden olevan manipuloimatonta, ennustamatonta, sekä julkisesti todennettavaa.

Konsensusmekanismeihin soveltuvat hajautetut satunnaisuuden tuottamisen protokollat käyttävät kryptografisia menetelmiä satunnaisuuden vaatimusten täyttämiseksi. Protokollat pohjautuvat käyttäjien syöttämään satunnaisuuteen. Käyttäjän syöttäessä satunnaisuutta viimeinen oman satunnaisuuden syöttänyt voi manipuloida satunnaisuutta. Kryptografiset menetelmät estävät erityisesti viimeisen paljastajan manipuloimisyritykset.

Vaatimalla käyttäjän lähettämään ensin satunnaisen syötteensä tiivisteen ennen syötteen paljastamista käyttäjä pakotetaan syöttämään valitsemansa syöte ilman, että käyttäjä voi valita syötettä muiden käyttäjien syötteiden perusteella. Julkinen todennettava salaisuuksien jakaminen (PVSS) \cite{StadlerMarkus2001PVSS} mahdollistaa sen, että vain rehelliset käyttäjät voivat paljastaa sellaisten käyttäjien satunnaisen syötteen, jotka eivät halua paljastaa syötettään yrittäen pysäyttää protokolla ja manipuloida tuotettavaa satunnaisuutta. PVSS olettaa, että vähintään $1/2$ käyttäjistä on rehellisiä. Homomorfisen salauksen avulla voidaan kehittää satunnaisuutta tuottava protokolla, jossa myös protokollan ulkopuolisten on mahdollista syöttää satunnaisuutta \cite{cherniaeva2019homomorphic}. Todennettavan viivefunktion (VDF) \cite{boneh_verifiable_2018} avulla käyttäjien syötteisiin voidaan asettaa sellainen viive, että viimeisen syötteen tarjoajan ei ole mahdollista simuloida satunnaista arvoa. Viivefunktion käytännön toteutus on kuitenkin haastava käyttäjien laitteistoerojen takia.

PoW-ketjussa toimiva älysopimus voi yksinkertaisimmillaan käyttää lohkon tiivistettä satunnaisena arvona. Kaivaja voi kuitenkin vaikuttaa lohkon tiivisteen jakaumaan jättämällä epäsuotuisan lohkon julkaisematta. Tämä hyökkäys kannattaa vain jos hyökkääjä voittaa manipuloinnilla enemmän kuin se voi saada palkinnoksi lohkon löytämisestä. Älysopimuksen yleisin turvallinen tapa satunnaisuuden saamiseen on käyttää Chainlink VRF \cite{noauthor_chainlink_nodate} palvelua. Tällöin älysopimus tilaa satunnaisen arvon, jonka Chainlink verkon satunnaislukuoraakkeli toimittaa älysopimukseen käyttäen Todennettavaa Satunnaisfunktiota (Verifiable Random Function, VRF) \cite{micali_verifiable_1999}, mikä hyödyntää tiivistefunktiota ja asymmetristä avainparia. Satunnaisesta arvosta on todennettavissa, että se on laskettu käyttäen tilauksen tietoja, sekä toimittajan salaista avainta. Tilauksen tietojen ja toimittajan salaisen avaimen käyttäminen mahdollistaa sen, että tilaaja tai toimittaja ei voi ennustaa arvoa.









\bibliographystyle{IEEEtran}
\bibliography{bibliography}
\end{document}