\chapter{Protokollat\label{results}}
\section{Yksinkertaiset metodit}

\subsection{Lohkon tiivisteen käyttäminen}
Yleisin lohkoketjusovelluksien käyttämä satunnaisuuden lähde on nykyisen lohkon hajautusarvo. Esimerkiksi lottoarvonta voitaisiin suorittaa niin, että älysopimukseen on määritelty, että voitto jaetaan lohkonumeroltaan tietyn tulevan lohkon hajautusarvon perusteella. Hajautusarvoa käytetään niin PoW- kuin PoS-ketjuissa, vaikkakin PoS-ketjussa hajautusarvo on helposti validaattorin manipuloitavissa. PoW-ketjun lohkon hajautusarvo täyttää vaatimukset 1-3, mutta kaivajat voivat vaikuttaa arvoon, joten luku on manipuloitavissa, eikä täytä vaatimusta 4. Jos kaivajan saama hyöty epäedullisen hajautusarvon omaavan lohkon julkaisematta jättämisestä on suurempi kuin lohkon julkaisemisesta saatava palkinto, on kaivajalla intressi jättää lohko julkaisematta. Tämä aiheuttaa vinouman satunnaislukujen jakaumaan lohkontuottajalle hyödylliseen suuntaan. Esimerkiksi Bitcoin-ketjun tapauksessa hyökkääjä ei tarvitse yhden lohkon hajautusarvon jakauman merkittävään manipuloimiseen enemmistöä, tai edes merkittävää taloudellista panostusta suhteessa ketjun kokoon \cite{pierrot_malleability_2018}.

\subsection{Kommitointi ja paljastaminen}
Kommitoinnin ja paljastamisen protokollassa satunnaisluku generoidaan kahdessa osassa. Ensimmäisellä kierroksella jokainen osallistuja kommitoi ensin kaikkien nähtäväksi lohkoketjuun salatun satunnaisluvun. Toisella kierroksella osallistujat paljastavat yksi kerrallaan kommitoidunsa satunnaisluvun. Paljastetuista luvuista lasketaan XOR, minkä seurauksena saadaan satunnaisluku. Olettaen, että vähintään yksi osallistuja on rehellinen, tämä protokolla täyttää vaatimukset 1, 2 ja 3. Ongelmana protokollassa on, että viimeinen paljastaja näkee generoidun satunnaisluvun ja voi täten olla julkaisematta syötettään, jos generoitu luku ei ole suosiollinen.

\section{VDF:n hyödyntäminen}

Lohkon tiivistettä sekä osallistujien kommitointia käyttävien protokollien heikkouksia voidaan parantaa käyttämällä VDF:ää \cite{boneh_verifiable_2018}. 

Tiivisteen käyttämisen tapauksessa älysopimuksessa määritettään, että satunnaisuuden lähteenä käytetään tietyn tulevaisuuden lohkon tiivistettä. Tiivistettä ei käytetä suoraan, vaan satunnainen arvo saadaan syöttämällä tiiviste VDF:lle. Kun VDF:n aiheuttama viive valitaan sopivasti, ei lohkon kaivaja ehdi simuloimaan satunnaista arvoa ja tekemään päätöstä lohkon julkaisemisesta ennen kuin rehelliset kaivajat ovat lisänneet uuden tai jopa monia uusia lohkoja lohkoketjuun.

Käytettäessä tiivistettä VDF:n kanssa on kommunikontikustannus vakio. Satunnaisluvun saamiseksi sopimuksen kanssa pitää vuorovaikuttaa vain kaksi kertaa. Kerran tulevan lohkon numeron päättämiseksi ja toisen kerran VDF:stä saadun tuloksen julkaisemiseksi ketjuun. Satunnaisluku on satunnainen riippumatta osallistujista.

Parannellussa kommitoinnin protokollassa kaikki osallistujat syöttävät sopimukseen heti satunnaisen arvonsa. Arvojen XOR tai tiiviste syötetään VDF:lle. Kuka tahansa osallistujista voi julkaista VDF:n tuottaman satunnaisen arvon. Myös tässä tapauksessa VDF aiheuttaa sen, että hyökkääjä ei voi simuloida satunnaista arvoa ennen kuin on myöhäistä. 

Kommunikointikustannus on myös vakio. Protokolla vaatii osallistujilta yhden transaktion satunnaisuuden kommitointia varten ja yhdeltä osallistujalta transaktion VDF:n tuottaman arvon julkaisemiseksi. Arvo on satunnainen kunhan vähintään yksi osallistuja on rehellinen.

VDF:n käyttämistä sovelluksissa vaikeuttaa käyttäjien ja mahdollisten hyökkääjien suuret erot laitteistojen tehokkuudessa. Viive tulisi voida valita niin, että protokolla on turvallinen tehokkaimpia laitteita vastaan, mutta kuitenkin niin, että rehellinen käyttäjä voi evaluoida funktion sovelluksen käyttötarkoituksen mukaisessa ajassa.

%\section{PVSS hyödyntäminen}
%Yksi tapa estää viimeisen paljastajan manipulointi on käyttää salaisuuden jakamista. (t, n)-salaisuuden jakaminen (SS) on tekniikka, jolla salaisuuden s jakaja voi jakaa n osallistujalle salaisuuden, niin, että mikä tahansa t osallistujan joukko voi rekonstruktuoida salaisuuden kun toisaalta mikä tahansa joukko kooltaan pienempi kuin t ei saa minkäänlaista tietoa salaisuudesta\cite{syta_scalable_2017}. Julkisesti todennettavassa salaisuuden jakamisessa (VSS) jakajan jakamat salaisuuden osuudet on todennettavissa valideiksi paljastamatta osuuksia tai salaisuutta. Esimerkkinä VSS käytöstä kuvataan tiivistetysti RandShare-protokolla\cite{syta_scalable_2017}. Käytettäessä julkisesti todennettavissa olevaa (t, n)-salaisuuden jakamista generointiprotokollassa tehdään oletus, että $N = 3t+1$ osallistujasta enintään t-1 osallistujaa on epärehellisiä. Tällöin protokollassa jokainen osallistuja jakaa ensin oman salaisen syötteensä t osaan VSS metodia käyttäen, ja lähettää osat muille osallistujille. Kun kaikilla osallistujilla on muiden osallistujien osat ja ne on vahvistettu, julkaistaan osat kaikille osallistujille. Osallistuja voi rekonstruktuoida toisen osallistujan salaisuuden jos hän on vastaanottanut vähintään t osaa. Siis epärehellinen osallistuja ei voi estää salaisuutensa julkaisemista, sillä oletuksen mukaan rehellisiä osallistujia on $2t+2$, mikä riittää siihen, että rehellisten osallistujien julkaistessa osansa, on rehellisillä osallistujilla vähintään t osaa jokaisesta salaisuudesta ja kaikki salaisuudet julkaistaan. 

%Tämä protokolla täyttää oletuksien rajoissa kriteerit 1-4, mutta esimerkiksi kommunikaatiokompleksisuus on $O(n^3)$\cite{syta_scalable_2017}, mikä on liian suuri moneen lohkoketjusovellukseen.

%\section{Muita protokollia}
%Muita mainitsemisen arvoisia protokollia ovat peliteoreettiset sekä homomorfista salausta käyttävät protokollat \cite{simic_review_2020}. 

%\section{Merlin-ketju (Merlin chain)}

%Merlin-ketju on sekvenssi $V_1, V_2 ... , V_n$, missä arvo $V_x$ on arvon $V_{x+1}$ hajautusarvo, jota voidaan käyttää lohkon hajautusarvojen kanssa vaatimukset 1-4 täyttävän protokollan rakentamiseen \cite{MerlinChains}.