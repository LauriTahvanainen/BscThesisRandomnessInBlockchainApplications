\chapter{Protokollat\label{results}}
\section{Yksinkertaiset metodit}

\subsection{Lohkon tiivisteen käyttäminen}
Yleisin lohkoketjusovelluksien käyttämä satunnaisuuden lähde on nykyisen lohkon hajautusarvo. Esimerkiksi lottoarvonta voitaisiin suorittaa niin, että älysopimukseen on määritelty, että voitto jaetaan lohkonumeroltaan tietyn tulevan lohkon hajautusarvon perusteella. Hajautusarvoa käytetään niin PoW- kuin PoS-ketjuissa, vaikkakin PoS-ketjussa hajautusarvo on helposti validaattorin manipuloitavissa. PoW-ketjun lohkon hajautusarvo täyttää vaatimukset 1-3, mutta kaivajat voivat vaikuttaa arvoon, joten luku on manipuloitavissa, eikä täytä vaatimusta 4. Jos kaivajan saama hyöty epäedullisen hajautusarvon omaavan lohkon julkaisematta jättämisestä on suurempi kuin lohkon julkaisemisesta saatava palkinto, on kaivajalla intressi jättää lohko julkaisematta. Tämä aiheuttaa vinouman satunnaislukujen jakaumaan lohkontuottajalle hyödylliseen suuntaan. Esimerkiksi Bitcoin-ketjun tapauksessa hyökkääjä ei tarvitse yhden lohkon hajautusarvon jakauman merkittävään manipuloimiseen enemmistöä, tai edes merkittävää taloudellista panostusta suhteessa ketjun kokoon \cite{pierrot_malleability_2018}.

\subsection{Kommitointi ja paljastaminen}
Kommitoinnin ja paljastamisen protokollassa satunnaisluku generoidaan kahdessa osassa. Ensimmäisellä kierroksella jokainen osallistuja kommitoi ensin kaikkien nähtäväksi lohkoketjuun satunnaisluvun tiivisteen. Toisella kierroksella osallistujat paljastavat yksi kerrallaan kommitoidunsa satunnaisluvun, joka vastaa ensin kommitoitua tiivistettä. Paljastetuista luvuista lasketaan XOR, minkä seurauksena saadaan satunnaisluku. Olettaen, että vähintään yksi osallistuja on rehellinen, tämä protokolla täyttää vaatimukset 1, 2 ja 3. Ongelmana protokollassa on, että viimeinen paljastaja näkee generoidun satunnaisluvun ja voi täten olla julkaisematta syötettään, jos generoitu luku ei ole suosiollinen.

\section{VDF:n hyödyntäminen}

Lohkon tiivistettä sekä osallistujien kommitointia käyttävien protokollien heikkouksia voidaan parantaa käyttämällä VDF:ää \cite{boneh_verifiable_2018}. 

Tiivisteen käyttämisen tapauksessa älysopimuksessa määritettään, että satunnaisuuden lähteenä käytetään tietyn tulevaisuuden lohkon tiivistettä. Tiivistettä ei käytetä suoraan, vaan satunnainen arvo saadaan syöttämällä tiiviste VDF:lle. Kun VDF:n aiheuttama viive valitaan sopivasti, ei lohkon kaivaja ehdi simuloimaan satunnaista arvoa ja tekemään päätöstä lohkon julkaisemisesta ennen kuin rehelliset kaivajat ovat lisänneet uuden tai jopa monia uusia lohkoja lohkoketjuun.

Käytettäessä tiivistettä VDF:n kanssa on kommunikontikustannus vakio. Satunnaisluvun saamiseksi sopimuksen kanssa pitää vuorovaikuttaa vain kaksi kertaa. Kerran tulevan lohkon numeron päättämiseksi ja toisen kerran VDF:stä saadun tuloksen julkaisemiseksi ketjuun. Satunnaisluku on satunnainen riippumatta osallistujista.

Parannellussa kommitoinnin protokollassa kaikki osallistujat syöttävät sopimukseen heti satunnaisen arvonsa. Arvojen XOR tai tiiviste syötetään VDF:lle. Kuka tahansa osallistujista voi julkaista VDF:n tuottaman satunnaisen arvon. Myös tässä tapauksessa VDF aiheuttaa sen, että hyökkääjä ei voi simuloida satunnaista arvoa ennen kuin on myöhäistä. 

Kommunikointikustannus on myös vakio. Protokolla vaatii osallistujilta yhden transaktion satunnaisuuden kommitointia varten ja yhdeltä osallistujalta transaktion VDF:n tuottaman arvon julkaisemiseksi. Arvo on satunnainen kunhan vähintään yksi osallistuja on rehellinen.

%\section{PVSS hyödyntäminen}
%Yksi tapa estää viimeisen paljastajan manipulointi on käyttää (t, n)-salaisuuden jakamista. (t, n)-salaisuuden jakaminen (SS) on tekniikka, jolla salaisuuden s jakaja voi jakaa n osallistujalle salaisuuden, niin, että mikä tahansa t osallistujan joukko voi rekonstruktuoida salaisuuden kun toisaalta mikä tahansa joukko kooltaan pienempi kuin t ei saa minkäänlaista tietoa salaisuudesta\cite{syta_scalable_2017}. Julkisesti todennettavassa salaisuuden jakamisessa (VSS) jakajan jakamat salaisuuden osuudet on todennettavissa valideiksi paljastamatta osuuksia tai salaisuutta. Esimerkkinä VSS käytöstä kuvataan tiivistetysti RandShare-protokolla\cite{syta_scalable_2017}. Käytettäessä julkisesti todennettavissa olevaa (t, n)-salaisuuden jakamista generointiprotokollassa tehdään oletus, että $N = 3t+1$ osallistujasta enintään t-1 osallistujaa on epärehellisiä. Tällöin protokollassa jokainen osallistuja jakaa ensin oman salaisen syötteensä t osaan VSS metodia käyttäen, ja lähettää osat muille osallistujille. Kun kaikilla osallistujilla on muiden osallistujien osat ja ne on vahvistettu, julkaistaan osat kaikille osallistujille. Osallistuja voi rekonstruktuoida toisen osallistujan salaisuuden jos hän on vastaanottanut vähintään t osaa. Siis epärehellinen osallistuja ei voi estää salaisuutensa julkaisemista, sillä oletuksen mukaan rehellisiä osallistujia on $2t+2$, mikä riittää siihen, että rehellisten osallistujien julkaistessa osansa, on rehellisillä osallistujilla vähintään t osaa jokaisesta salaisuudesta ja kaikki salaisuudet julkaistaan. 

%Tämä protokolla täyttää oletuksien rajoissa kriteerit 1-4, mutta esimerkiksi kommunikaatiokompleksisuus on $O(n^3)$\cite{syta_scalable_2017}, mikä on liian suuri moneen lohkoketjusovellukseen.

%\section{Muita protokollia}
%Muita mainitsemisen arvoisia protokollia ovat peliteoreettiset sekä homomorfista salausta käyttävät protokollat \cite{simic_review_2020}. 

\section{Merlin-ketju (Merlin chain)}

Merlin-ketju on sekvenssi $V_1, V_2 ... , V_n$, missä arvo $V_x$ on arvon $V_{x+1}$ hajautusarvo, jota voidaan käyttää lohkon hajautusarvojen kanssa vaatimukset 1-4 täyttävän protokollan rakentamiseen \cite{MerlinChains}. Protokolla estää kaivajien sekä osallistujien yritykset vaikuttaa tuotettavaan satunnaislukuun ja se pyrkii toimimaan jatkuvana satunnaislukujen lähteenä, majakkana. 

Ennen osallistumista jokainen osallistuja generoi satunnaisen arvon $V_n$, tallentaa tämän ja generoi arvosta Merlin-ketjun. $n$ valitaan niin, että Merlin-ketjussa on niin monta arvoa kuin osallistuja ikinä tarvitsee protokollan suorittamiseen tulevaisuudessa. Osallistuja paljastaa ketjun arvoja alkaen arvosta $V_1$. Paljastettuaan arvon $V_x$ osallistujan on pakko paljastaa seuraavaksi arvo $V_{x+1}$ tai löytää törmäys tiivisteelle $V_x$, minkä oletetaan olevan laskennallisesti mahdotonta. Merlin-ketju pakottaa osallistujan käyttämään ketjun ennalta määräämiä, mutta kuitenkin muille osallistujille satunnaisia, arvoja. Täten osallistuja ei voi reagoida muiden osallistujien arvoihin vaihtamalla omaa arvoaan.  

Olkoon seuraavan tuotettava satunnainen arvo $R_x$. Edellinen arvo on $R_{x-1}$ ja edellisen arvon lohko $B(R_{x-1})$
Yksinkertaisessa mallissa on yksi osallistuja, tuottaja, joka syöttää älysopimukselle Merlin-ketjun arvoja yhdessä aikaleiman kanssa. Arvo hyväksytään sen täyttäessä tietyt ehdot. Arvon on oltava validi Merlin ketjun arvo, eli $H(V_x) = V_{x-1}$. Nykyisen lohkon lohkonumeron on oltava niin paljon suurempi kuin lohkon, jossa tuotettiin edellinen satunnaisluku, että voidaan olla varmoja edellisen lohkon muuttumattomuudesta. Sopimus tuottaa satunnaisen arvon laskemalla tiivisteen osallistujan arvosta, sekä edellisen satunnaisluvun lohkon seuraajan tiivisteestä, $R_x = H(\, V_x || H(B(R_{x-1})+1) \,)$. Satunnaisen arvon yhteydessä sopimus tuottaa myös aikaleiman, joka on aika, jolloin tuottaja pystyi ensimmäisen kerran simuloimaan tuotettavan arvon. 

Protokolla estää kaivajia manipuloimasta satunnaista arvoa, sillä kaivajilla ei ole tiedossa arvo $V_x$ lohkon kaivamisen hetkellä, eikä arvoa siten voi simuloida. Protokolla estää tuottajan tekemät manipuloimisyritykset käyttämällä Merlin-ketjua ja täten pakottamalla tuottajan syöttämään ennaltamäärättyjä arvoja. Vaatimukset 1-4 täyttyvät, kunhan tuottaja ei paljasta Merlin-ketjua etukäteen kaivajille. 

Koostamalla protokollan monesta tuottajasta sivuutetaan tuottajien ja kaivajien yhdessä tapahtuvan manipuloinnin uhka. Tällöin $R_x$ koostuu $N$ tuottajan satunnaisesta arvosta. Tuottajat suorittavat yhden tuottajan protokollan kerran ja jäävät odottamaan kunnes kaikki tuottajat ovat tuottaneet satunnaisen arvon. Tuottajien satunnaisista arvoista otetaan XOR, mikä muodostaa protokollan lopullisen satunnaisen arvon. Protokolla täyttää vaatimukset, kunhan vähintään yksi tuottaja toimii rehellisesti.

Yhden ja monen tuottajan protokollan toteuttamisessa on ongelmana, se mitä tehdään tuottajille, jotka eivät paljasta Merlin-ketjun arvojaan. Paljastamatta jättäminen lukitsee koko protokollan suorituksen. Tuottajaa vaaditaan pitämään Merlin-ketjun arvo $V_n$, jolla voi palauttaa koko ketjun, joten muut tuottajat voivat selvästi vaatia tuottajalta protokollan mukaista toimintaa. Tällaisia tuottajia ei voi sivuuttaa tietyn ajan jälkeen, sillä tällöin tuottaja voi vaikuttaa päätöksellään arvoon. Mell et al. ehdottavat, että arvonsa salaavat tuottajat voi poistaa tuottajien joukosta ja tuottajan Merlin-ketjun voisi palauttaa ratkaisemalla vähintään tietyn ajan vaativan ongelman (Timelock-puzzle) \cite{MerlinChains}. Jos tuottaja vain poistetaan, ilman tämän ketjun palauttamista, kohtaa protokolla yhä ongelman, sillä lohkoketjuympäristössä hyökkääjä voi luoda uuden tilin uudella Merlin-ketjulla, jolla osallistua taas tuottajana. Hyökkääjä lopettaa arvojen julkaiseminen aina kohdatessaan epäsuotuisen arvon ja pystyy täten vääristämään lopullisen satunnaisen arvon jakaumaa. Hyökkääjän palautettua Merlin-ketjua voitaisiin käyttää vain protokollan nykyisen kierroksen loppuunsaattamiseen.

Protokollan pysähtymisen ongelmaa ei voi myöskään ratkaista tekemällä tuottajien osallistumisen avoimeksi, tarkoittaen, että esimerkiksi yhdellä protokollan kierroksella vaaditaan $N$ kenen tahansa lohkoketjun käyttäjän Merlin-ketjusta tuotettu satunnaisluku. Tällöin menetetään Merlin-ketjun tuoma etu, sillä hyökkääjä voi halutessaan hallinnoida tuhansia tilejä ja Merlin-ketjuja, joista valita sopiva arvo. Protokolla luhistuu tavalliseksi kommitointiprotokollaksi.

Muina ratkaisuina, tuottajat voisivat esimerkiksi tallettaa älysopimukseen vakuutena toiminnastaa kryptovaluuttaa. Jos tuottaja ei paljasta arvoaan tietyssä ajassa, menettää hän vakuutensa. Uusien tuottajien luominen olisi tällöin taloudellisesti kallista. Tuottajien historiaa voidaan myös seurata ja luoda avoin tilastointi tuottajan maineelle. Maineikkaita tuottajia palkittaisiin protokollan mukaisesta toiminnasta. 

