\chapter{Protokollat\label{results}}
\section{Yksinkertaiset metodit}

\subsection{Hajautusarvo}
Yleisin lohkoketjusovelluksien käyttämä satunnaisuuden lähde on nykyisen lohkon hajautusarvo. Esimerkiksi lottoarvonta voitaisiin suorittaa niin, että älysopimukseen on määritelty, että voitto jaetaan lohkonumeroltaan tietyn tulevan lohkon hajautusarvon perusteella. Hajautusarvoa käytetään niin PoW- kuin PoS-ketjuissa, vaikkakin PoS-ketjussa hajautusarvo on helposti validaattorin manipuloitavissa. PoW-ketjun lohkon hajautusarvo täyttää vaatimukset 1-3, mutta kaivajat voivat vaikuttaa arvoon, joten luku on manipuloitavissa, eikä täytä vaatimusta 4. Jos kaivajan saama hyöty epäedullisen hajautusarvon omaavan lohkon julkaisematta jättämisestä on suurempi kuin lohkon julkaisemisesta saatava palkinto, on kaivajalla intressi jättää lohko julkaisematta. Tämä aiheuttaa vinouman satunnaislukujen jakaumaan lohkontuottajalle hyödylliseen suuntaan. Esimerkiksi Bitcoin-ketjun tapauksessa hyökkääjä ei tarvitse yhden lohkon hajautusarvon jakauman merkittävään manipuloimiseen enemmistöä, tai edes merkittävää taloudellista panostusta suhteessa ketjun kokoon \cite{pierrot_malleability_2018}.

\section{Kommitointi ja paljastaminen}
Kommitoinnin ja paljastamisen protokollassa satunnaisluku generoidaan kahdessa osassa. Ensimmäisellä kierroksella jokainen osallistuja kommitoi ensin kaikkien nähtäväksi salatun satunnaisluvun. Toisella kierroksella osallistujat paljastavat yksi kerrallaan kommitoidunsa satunnaisluvun. Paljastetut satunnaisluvut yhdistetään, minkä seurauksena saadaan satunnaisluku. Olettaen, että vähintään yksi osallistuja on rehellinen, tämä protokolla täyttää vaatimukset 2, 3 ja 4 \cite{simic_review_2020}. Ongelmana protokollassa on, että viimeinen paljastaja näkee generoidun satunnaisluvun ja voi täten olla julkaisematta syötettään, jos generoitu luku ei ole suosiollinen \cite{simic_review_2020}.

Yksi tapa estää viimeisen paljastajan manipulointi on käyttää salaisuuden jakamista. (t, n)-salaisuuden jakaminen (SS) on tekniikka, jolla salaisuuden s jakaja voi jakaa n osallistujalle salaisuuden, niin, että mikä tahansa t osallistujan joukko voi rekonstruktuoida salaisuuden kun toisaalta mikä tahansa joukko kooltaan pienempi kuin t ei saa minkäänlaista tietoa salaisuudesta\cite{syta_scalable_2017}. Julkisesti todennettavassa salaisuuden jakamisessa (VSS) jakajan jakamat salaisuuden osuudet on todennettavissa valideiksi paljastamatta osuuksia tai salaisuutta. Esimerkkinä VSS käytöstä kuvataan tiivistetysti RandShare-protokolla\cite{syta_scalable_2017}. Käytettäessä julkisesti todennettavissa olevaa (t, n)-salaisuuden jakamista generointiprotokollassa tehdään oletus, että $N = 3t+1$ osallistujasta enintään t-1 osallistujaa on epärehellisiä. Tällöin protokollassa jokainen osallistuja jakaa ensin oman salaisen syötteensä t osaan VSS metodia käyttäen, ja lähettää osat muille osallistujille. Kun kaikilla osallistujilla on muiden osallistujien osat ja ne on vahvistettu, julkaistaan osat kaikille osallistujille. Osallistuja voi rekonstruktuoida toisen osallistujan salaisuuden jos hän on vastaanottanut vähintään t osaa. Siis epärehellinen osallistuja ei voi estää salaisuutensa julkaisemista, sillä oletuksen mukaan rehellisiä osallistujia on $2t+2$, mikä riittää siihen, että rehellisten osallistujien julkaistessa osansa, on rehellisillä osallistujilla vähintään t osaa jokaisesta salaisuudesta ja kaikki salaisuudet julkaistaan. 

Tämä protokolla täyttää oletuksien rajoissa kriteerit 1-4, mutta esimerkiksi kommunikaatiokompleksisuus on $O(n^3)$\cite{syta_scalable_2017}, mikä on liian suuri moneen lohkoketjusovellukseen.

\section{Peräkkäinen työtodistus(Proof of Work)}
Työtodistuksella tarkoitetaan arvoa, josta voidaan (tehokkaasti) todistaa, että arvon hankkimisen eteen on tehty tietty määrä työtä. Esitellään peräkkäistä työtodistusta käyttävä satunnaislukujen generoimisen protokolla\cite{lesaege_kleros_2020}. Jokainen osallistuja syöttää ensin yhteiseen siemenarvoon oman arvonsa. Sitten jokainen osallistuja laskee rekursiivisesti hajautusarvon tästä siemenarvosta h kertaa, tallentaen välituloksia. $h$ on ennalta sovittu vaikeusaste. Viimeinen hajautusarvo h toimii tässä tapauksena työtodistuksena. Tätä laskutoimitusta ei voi rinnakkaistaa, ja sille on jokin teoreettinen ajallinen alaraja. Ensimmäinen viimeisen hajautusarvon laskija julkaisee arvon osallistujille, minkä jälkeen arvon oikeellisuus voidaan todentaa muiden toimesta vertaamalla laskijan tuloksia binäärihaulla muiden osallistujien tuloksiin. Todistuksen oikeellisuus voidaan todentaa $log(h)$ kyselyllä. Kun arvo on todennettu, voidaan sitä käyttää satunnaisena arvona.

Protokolla täyttää vaatimukset 1-7. Ongelmana on, että vaikeustason h pitää olla riittävän suuri, jotta kaikki vaatimukset täyttyvät, jolloin generointiin kuluva aika t kasvaa monille käytännön sovelluksille liian suureksi. Osallistujien laitteiston suorituskykyjen erot voivat aiheuttaa ongelmia, sillä hajautusarvon laskeminen on laitteistotasolla helposti optimoitavissa.

Äsken esitelty rekursiivinen funktio on yksinkertainen esimerkki todennettavasta viivefunktiosta (Verifiable Delay Function, VDF). Satunnaislukujen generoimiseen sopivan VDF:n pitäisi aiheuttaa riittävä viive, jotta vaatimukset 1-4 täyttyvät, mutta käytännön sovelluksia varten vahvistamisen pitäisi onnistua mahdollisimman pienessä ajassa. Funktion laskennan laiteoptimoinnista saatavan hyödyn pitäisi olla mahdollisimman pieni. VDF:ää voi käyttää kommitoinnin ja paljastamisen protokollan osana poistamaan viimeisen paljastajan etu \cite{boneh_verifiable_2018}.

\section{Muita protokollia}
Muita mainitsemisen arvoisia protokollia ovat peliteoreettiset sekä homomorfista salausta käyttävät protokollat \cite{simic_review_2020}. 