\chapter{Satunnaisuuden tuottaminen\label{results}}

\section{Ulkoiset metodit}

Lohkoketjusovelluksen ei tarvitse itse tuottaa satunnaisuuttaan. Esitetään kaksi yleisintä tapaa, jolla älysopimukset voivat saada käyttöönsä satunnaisuutta.

Yleisin lohkoketjusovelluksien käyttämä satunnaisuuden lähde on nykyisen lohkon hajautusarvo. Esimerkiksi lottoarvonta voitaisiin suorittaa niin, että älysopimukseen on määritelty, että voitto jaetaan lohkonumeroltaan tietyn tulevan lohkon hajautusarvon perusteella. Hajautusarvoa käytetään niin PoW- kuin PoS-ketjuissa, vaikkakin PoS-ketjussa hajautusarvo on helposti validaattorin manipuloitavissa. PoW-ketjun lohkon hajautusarvo täyttää vaatimukset 1-3, mutta kaivajat voivat vaikuttaa arvoon, joten luku on manipuloitavissa eikä täytä vaatimusta 4. Jos kaivajan saama hyöty epäedullisen hajautusarvon omaavan lohkon julkaisematta jättämisestä on suurempi kuin lohkon julkaisemisesta saatava palkinto, on kaivajalla intressi jättää lohko julkaisematta. Tämä aiheuttaa vinouman satunnaislukujen jakaumaan lohkontuottajalle hyödylliseen suuntaan. Esimerkiksi Bitcoin-ketjun tapauksessa hyökkääjä ei tarvitse yhden lohkon hajautusarvon jakauman merkittävään manipuloimiseen enemmistöä tai edes merkittävää taloudellista panostusta suhteessa ketjun kokoon \cite{pierrot_malleability_2018}.

Toinen suosittu ulkoinen lähde on Chainlink VRF \cite{noauthor_verifiable_2020}. Tällöin sovellus käyttää Chainlink verkon satunnaislukuoraakkelia, joka tarjoaa sovellukselle vaatimukset 1-4 täyttävän satunnaisluvun käyttäen todennettavaa satunnaisfunktiota (VRF). Älysopimuksen pitää pyytää satunnaislukua, minkä jälkeen Chainlink verkon serveri lähettää älysopimukseen satunnaisen arvon sisältävän transaktion. Tämän satunnaisluvun saaminen kuitenkin maksaa Chainlink virtuaalivaluuttaa, ja monelle sovellukselle ulkopuolisesta palvelusta maksaminen ei ole käytännöllistä. Jos Chainlink VRF:n käyttöä ei toteuteta oikein älysopimuksessa, niin esimerkiksi kaivajilla voi olla mahdollisuus vaikuttaa arvoon \cite{noauthor_vrf_nodate}.

Lohkoketjusovellus voi vaatia käytettävän protokollaa, jossa satunnaisluvun tuottaminen onnistuu itse sovelluksen käyttäjiltä. Seuraavaksi esitetään protokollia, joissa sovelluksen käyttäjät vastaavat satunnaisuuden tuottamisesta.

\section{Kommitointi ja paljastaminen} %Miksi täyttää vaatimjukset?
Kommitoinnin ja paljastamisen protokollassa satunnaisluku generoidaan kahdessa osassa. Ensimmäisellä kierroksella jokainen osallistuja kommitoi ensin kaikkien nähtäväksi lohkoketjuun satunnaisluvun tiivisteen. Toisella kierroksella osallistujat paljastavat yksi kerrallaan kommitoidunsa satunnaisluvun, joka vastaa ensin kommitoitua tiivistettä. Paljastetuista luvuista lasketaan XOR, minkä seurauksena saadaan satunnaisluku. Olettaen, että vähintään yksi osallistuja on rehellinen, tämä protokolla täyttää vaatimukset 1, 2 ja 3. Ongelmana protokollassa on, että viimeinen paljastaja näkee generoidun satunnaisluvun ja voi täten olla julkaisematta syötettään jos generoitu luku ei ole suosiollinen. 

\section{VDF:n hyödyntäminen}
Lohkon tiivistettä sekä osallistujien kommitointia käyttävien protokollien heikkouksia voidaan parantaa käyttämällä VDF:ää \cite{boneh_verifiable_2018}. 

Tiivisteen käyttämisen tapauksessa älysopimuksessa määritetään, että satunnaisuuden lähteenä käytetään tietyn tulevaisuuden lohkon tiivistettä. Tiivistettä ei käytetä suoraan, vaan satunnainen arvo saadaan syöttämällä tiiviste VDF:lle. Kun VDF:n aiheuttama viive valitaan sopivasti, ei lohkon kaivaja ehdi simuloimaan satunnaista arvoa ja tekemään päätöstä lohkon julkaisemisesta ennen kuin rehelliset kaivajat ovat lisänneet uuden tai jopa monia uusia lohkoja lohkoketjuun.

Käytettäessä tiivistettä VDF:n kanssa on kommunikointikustannus vakio. Satunnaisluvun saamiseksi sopimuksen kanssa pitää vuorovaikuttaa vain kaksi kertaa. Kerran tulevan lohkon numeron päättämiseksi ja toisen kerran VDF:stä saadun tuloksen julkaisemiseksi ketjuun. Satunnaisluku on satunnainen riippumatta osallistujista.

Parannellussa kommitoinnin protokollassa kaikki osallistujat syöttävät sopimukseen heti satunnaisen arvonsa. Arvojen XOR tai tiiviste syötetään VDF:lle. Kuka tahansa osallistujista voi julkaista VDF:n tuottaman satunnaisen arvon. Myös tässä tapauksessa VDF aiheuttaa sen, että hyökkääjä ei voi simuloida satunnaista arvoa ennen kuin on myöhäistä. 

Kommunikointikustannus on myös vakio. Protokolla vaatii osallistujilta yhden transaktion satunnaisuuden kommitointia varten ja yhdeltä osallistujalta transaktion VDF:n tuottaman arvon julkaisemiseksi. Arvo on satunnainen kunhan vähintään yksi osallistuja on rehellinen.

\section{Julkisen todennettavan salaisuuksien jakamisen hyödyntäminen}

\textit{Salaisuuksien jakaminen Secret Sharing, SS)} on tekniikka, jolla salaisuuden s jakaja voi jakaa n osallistujalle salaisuuden niin, että mikä tahansa t osallistujan joukko voi rekonstruktuoida salaisuuden kun toisaalta mikä tahansa joukko kooltaan pienempi kuin t ei saa minkäänlaista tietoa salaisuudesta\cite{shamir_how_1979}. \textit{Julkisesti todennettavassa salaisuuden jakamisessa (PVSS)} on julkisesti todennettavissa, että jakaja on jakanut validit salaisuuden osat \cite{StadlerMarkus2001PVSS}. Todentamisen voi suorittaa myös ulkopuoliset, joille ei ole jaettu salaisuuksien osia. Todennettavassa salaisuuden jakamisessa (VSS) osien todentamisen voi suorittaa vain salaisuuden osan vastaanottaja.

Yksi tapa estää kommitoinnin ja paljastamisen protokollan viimeisen paljastajan manipulointi on käyttää VSS:ää, mikä poistaa hyökkääjältä mahdollisuuden olla julkaisematta syötettään. Esimerkkinä VSS käytöstä kuvataan tiivistetysti RandShare-protokolla\cite{syta_scalable_2017}. Protokollassa tehdään oletus, että $N = 3t+1$ osallistujasta enintään $t$ osallistujaa on epärehellisiä. Tällöin protokollassa jokainen osallistuja jakaa ensin oman salaisen syötteensä $n$ osaan VSS metodia käyttäen ja lähettää osat muille osallistujille. Salaisuuden rekonstruktuointi vaatii $t+1$ osaa. Kun kaikilla osallistujilla on muiden osallistujien osat ja ne on vahvistettu, ilmoitetaan valideista osista muille osallistujille. Validien salaisuuksien osat julkaistaan kaikille osallistujille vasta kun on tiedossa, että valideja salaisuuksia on vähintään $t+1$. Osallistuja voi rekonstruktuoida toisen osallistujan salaisuuden jos hän on vastaanottanut vähintään t osaa muilta osallistujilta. Siis epärehellinen osallistuja ei voi estää salaisuutensa julkaisemista, sillä oletuksen mukaan rehellisiä osallistujia on $2t+1$, mikä riittää siihen, että rehellisten osallistujien julkaistessa osansa, on rehellisillä osallistujilla vähintään $t$ osaa jokaisesta salaisuudesta ja kaikki salaisuudet julkaistaan. 

Tämä protokolla täyttää kriteerit 1-4 kun hyökkääjiä on enintään $t$, mutta esimerkiksi kommunikaatiokompleksisuus on $O(n^3)$ \cite{syta_scalable_2017}, mikä on liian suuri moneen lohkoketjusovellukseen. Lohkoketjusovelluksen tapauksessa protokollan on myös hyvä käyttää PVSS metodia, jotta kuka tahansa voi varmentaa protokollan oikeellisuuden. RandShare toimiikin vain yksinkertaisena esimerkkinä VSS:n käytöstä. 

%Kuinka viitata?
PVSS pohjautuvia protokollia satunnaisuuden tuottamiseen on tutkittu laajasti viime vuosina \cite{bhat2022optrand}. Usein protokollia ei ole tarkoitettu toteutetuksi lohkoketjusovelluksessa, vaan tutkimusta motivoi konsensusmekanismien johtajan valinnan tai julkisen satunnaisuuden majakan kehittäminen. Protokollia on pyritty optimoimaan eri tavoin ja viimeisin protokolla OptRand saavuttaa $O(N^2)$ pahimman tapauksen kommunikointikompleksisuuden \cite{bhat2022optrand} sietäen enintään $N/2$ hyökkääjää.

Vaikka kommunikointikompleksisuutta onkin saatu madallettua, ei $O(N^2)$ kompleksisuus ole vielä riittävän pieni, että PVSS-protokollaa voitaisiin käyttää lohkoketjusovelluksessa, jossa satunnaisuuden tuottamiseen osallistuu tuhansia osallistujia, sillä yksittäisen käyttäjän transaktiokulut kasvavat liian suuriksi. Jos PVSS-protokollana toteutettuun lohkoketjusovellukseen osallistuminen on ilmaista, niin jopa yksi hyökkääjä voi  %TODO SYBIL SELITYS

\section{Merlin-ketju (Merlin chain)}

Merlin-ketju on sekvenssi $V_1, V_2 ... , V_n$, missä arvo $V_x$ on arvon $V_{x+1}$ hajautusarvo, jota voidaan käyttää lohkon hajautusarvojen kanssa vaatimukset 1-4 täyttävän protokollan rakentamiseen \cite{MerlinChains}. Protokolla estää kaivajien sekä osallistujien yritykset vaikuttaa tuotettavaan satunnaislukuun ja se pyrkii toimimaan jatkuvana satunnaislukujen lähteenä, majakkana. 

Ennen osallistumista jokainen osallistuja generoi satunnaisen arvon $V_n$, tallentaa tämän ja generoi arvosta Merlin-ketjun. $n$ valitaan niin, että Merlin-ketjussa on niin monta arvoa kuin osallistuja ikinä tarvitsee protokollan suorittamiseen tulevaisuudessa. Osallistuja paljastaa ketjun arvoja alkaen arvosta $V_1$. Paljastettuaan arvon $V_x$ osallistujan on pakko paljastaa seuraavaksi arvo $V_{x+1}$ tai löytää törmäys tiivisteelle $V_x$, minkä oletetaan olevan laskennallisesti mahdotonta. Merlin-ketju pakottaa osallistujan käyttämään ketjun ennalta määräämiä, mutta kuitenkin muille osallistujille satunnaisia, arvoja. Täten osallistuja ei voi reagoida muiden osallistujien arvoihin vaihtamalla omaa arvoaan.  

Olkoon seuraavana tuotettava satunnainen arvo $R_x$. Edellinen arvo on $R_{x-1}$ ja edellisen arvon lohko $B(R_{x-1})$
Yksinkertaisessa mallissa on yksi osallistuja, tuottaja, joka syöttää älysopimukselle Merlin-ketjun arvoja yhdessä aikaleiman kanssa. Arvo hyväksytään sen täyttäessä tietyt ehdot. Arvon on oltava validi Merlin ketjun arvo, eli $H(V_x) = V_{x-1}$. Nykyisen lohkon lohkonumeron on oltava niin paljon suurempi kuin lohkon, jossa tuotettiin edellinen satunnaisluku, että voidaan olla varmoja edellisen lohkon muuttumattomuudesta. Sopimus tuottaa satunnaisen arvon laskemalla tiivisteen osallistujan arvosta, sekä edellisen satunnaisluvun lohkon seuraajan tiivisteestä, $R_x = H(\, V_x || H(B(R_{x-1})+1) \,)$. Satunnaisen arvon yhteydessä sopimus tuottaa myös aikaleiman, joka on aika, jolloin tuottaja pystyi ensimmäisen kerran simuloimaan tuotettavan arvon. 

Protokolla estää kaivajia manipuloimasta satunnaista arvoa, sillä kaivajilla ei ole tiedossa arvo $V_x$ lohkon kaivamisen hetkellä, eikä arvoa siten voi simuloida. Protokolla estää tuottajan tekemät manipuloimisyritykset käyttämällä Merlin-ketjua ja täten pakottamalla tuottajan syöttämään ennaltamäärättyjä arvoja. Vaatimukset 1-4 täyttyvät, kunhan tuottaja ei paljasta Merlin-ketjua etukäteen kaivajille. 

Koostamalla protokollan monesta tuottajasta sivuutetaan tuottajien ja kaivajien yhdessä tapahtuvan manipuloinnin uhka. Tällöin $R_x$ koostuu $N$ tuottajan satunnaisesta arvosta. Tuottajat suorittavat yhden tuottajan protokollan kerran ja jäävät odottamaan kunnes kaikki tuottajat ovat tuottaneet satunnaisen arvon. Tuottajien satunnaisista arvoista otetaan XOR, mikä muodostaa protokollan lopullisen satunnaisen arvon. Protokolla täyttää vaatimukset, kunhan vähintään yksi tuottaja toimii rehellisesti.

Yhden ja monen tuottajan protokollan toteuttamisessa on ongelmana, se mitä tehdään tuottajille, jotka eivät paljasta Merlin-ketjun arvojaan. Paljastamatta jättäminen lukitsee koko protokollan suorituksen. Tuottajaa vaaditaan pitämään Merlin-ketjun arvo $V_n$, jolla voi palauttaa koko ketjun, joten muut tuottajat voivat selvästi vaatia tuottajalta protokollan mukaista toimintaa. Tällaisia tuottajia ei voi sivuuttaa tietyn ajan jälkeen, sillä tällöin tuottaja voi vaikuttaa päätöksellään arvoon. Mell et al. ehdottavat, että arvonsa salaavat tuottajat voi poistaa tuottajien joukosta ja tuottajan Merlin-ketjun voisi palauttaa ratkaisemalla vähintään tietyn ajan vaativan ongelman (Timelock-puzzle) \cite{MerlinChains}. 

Jos tuottaja vain poistetaan, ilman tämän ketjun palauttamista, kohtaa protokolla yhä ongelman, sillä lohkoketjuympäristössä hyökkääjä voi luoda uuden tilin uudella Merlin-ketjulla, jolla osallistua taas tuottajana. Hyökkääjä lopettaa arvojen julkaiseminen aina kohdatessaan epäsuotuisen arvon ja pystyy täten vääristämään lopullisen satunnaisen arvon jakaumaa. Hyökkääjän palautettua Merlin-ketjua voitaisiin käyttää vain protokollan nykyisen kierroksen loppuunsaattamiseen.

Protokollan pysähtymisen ongelmaa ei voi myöskään ratkaista tekemällä tuottajien osallistumisen avoimeksi, tarkoittaen, että esimerkiksi yhdellä protokollan kierroksella vaaditaan $N$ kenen tahansa lohkoketjun käyttäjän Merlin-ketjusta tuotettu satunnaisluku. Tällöin menetetään Merlin-ketjun tuoma etu, sillä hyökkääjä voi halutessaan hallinnoida tuhansia tilejä ja Merlin-ketjuja, joista valita sopiva arvo ja protokolla luhistuu tavalliseksi kommitointiprotokollaksi.

Muina ratkaisuina, tuottajat voisivat esimerkiksi tallettaa älysopimukseen vakuutena toiminnastaa kryptovaluuttaa. Jos tuottaja ei paljasta arvoaan tietyssä ajassa, menettää hän vakuutensa. Uusien tuottajien luominen olisi tällöin taloudellisesti kallista. Tuottajien historiaa voidaan myös seurata ja luoda avoin tilastointi tuottajan maineelle. Maineikkaita tuottajia palkittaisiin protokollan mukaisesta toiminnasta. 

Caucus on toinen Merlin-ketjuja käyttävä protokolla \cite{DBLP:journals/corr/abs-1801-07965}. Merlin-ketjun sijasta kirjoittajat käyttävät nimeä tiivisteketju. Caucus yrittää ratkaista konsensusmekanismin johtajan valinnan ongelman tuottamalla satunnaisuutta, jonka mukaan johtaja valitaan. Protokolla vaatii satunnaisen arvon konfigurointia varten ja kirjoittajat esittävät, että tähän voidaan käyttää esimerkiksi PVSS-protokolla. Caucus on sovellettavissa PoS-lohkoketjun konsensusmekanismiin, mutta sitä voi käyttää myös lohkoketjusovelluksessa. Kirjoittajat esittävät älysopimustoteutuksen, joka kykeni kirjoitushetkellä 2018 tuottamaan yhden satunnaisluvun 0.1 dollarin kustannuksella. Kommunikointikompleksisuus on Merlin-ketju-protokollan tapaan O(1), mutta Caucus protokollan arvoja voi myös manipuloida. Valittu johtaja voi vääristää satunnaisuuden jakaumaa olemalla julkaisematta epäsuotuista tiivisteketjun arvoaan. Caucus jättää tarkoituksella kyseisen ongelman ratkaisun tulevalle tutkimukselle, mutta kirjoittavat esittävät, että manipuloinnin estämiseen voitaisiin käyttää yllä mainittua osallistujan vakuustalletusta.

Caucus-protokollan manipulointia voisi mahdollisesti ehkäistä käyttämällä VDF:ää. Johtaja julkaisisi tiivisteketjun arvonsa lohkoketjuun ja lopullinen satunnainen arvo saataisiin VDF:stä. Varsinaisen tuloksen voisi julkaista kuka tahansa. Tällöin valittu johtaja ei voisi simuloida satunnaista arvoa, eikä täten tehdä päätöstä julkaisemisesta.

Verrattuna Merlin-protokollaan, Caucus-protokollassa on pienempi riski protokollan pysähtymiselle siksi, että Merlin-protokollan tapauksessa satunnaisluvun generoimiseksi pitää odottaa kaikkia osallistujia, kun taas Caucus-protokollassa tuotettavana oleva satunnaisuus riippuu yhdestä osallistujasta, valitusta johtajasta. Merlin-protokolla on toteutettu älysopimuksena  \cite{MerlinChains}, kuten myös Caucus protokolla \cite{DBLP:journals/corr/abs-1801-07965}, mutta kirjoitushetkellä onnistuin löytämään vain Caucus-protokollan toteutuksen. Saatavilla oleva valmis toteutus helpottaa merkittävästi protokollan käyttöönottoa lohkoketjusovelluksessa.


% viimeisen julkaisijan ongelma?

% Julkaisija päättää, monta arvoa seuraavan on julkaistava? Täytyy julkaista tietty määrä arvoja?

\section{Homomorfinen salaus (Homomorphic Encryption, HE)}

\textit{Homomorfinen salaus} mahdollistaa tiettyjen laskutoimitusten suorittamisen salatulla datalla ilman, että salausta puretaan \cite{alma9928100443506253}. Laskutoimitusten tulos vastaa samaa kuin, että laskutoimitukset olisi tehty salaamattomalla datalla. Täysin homomorfisessa salauksessa (Fully Homomorphic Encryption, FHE) salatulla datalle voidaan suorittaa mitä tahansa laskentaa. Osittain homomorfinen salaus (Partially Homomorphic Encryption, PHE) mahdollistaa vain tiettyjen laskutoimitusten suorittamisen salatulle datalle.

HERB protokolla käyttää satunnaisuuden tuottamiseen osittain homomorfista salausta ja se tuottaa oletuksiensa rajoissa vaatimukset 1-4 täyttäviä satunnaislukuja \cite{cherniaeva2019homomorphic}. HERB olettaa, että osallistujia on $N=3t+1$, joista hyökkääjiä on enintään $t$. HERB konfiguroidaan hajautetulla avainten generoinnilla (Distributed Key Generation, DKG \cite{pedersen1991threshold}), jossa osallistujat generoivat asymmetrisen salauksen avainparin, jonka salainen avain on jaettu N osallistujalle ja salaisen avaimen käyttämiseen salauksen purkamiseksi vaaditaan $t+1$ avaimen osaa. Osallistujat kommitoivat satunnaisuutta salattuna käyttäen jaettua salaista avainta vastaavaa yhteistä julkista avainta. Vähintään $t+1$ salattua kommitointia lasketaan yhteen, minkä jälkeen $t+1$ osallistujaa purkaa yhteenlaskettujen kommitointien salauksen. Purettu yhteenlaskettu kommitointi toimii satunnaisena arvona. 

%Linkki toteutuksiin
DKG on olemassaoleva älysopimustoteutus, jossa yksittäisen käyttäjän kommunikaatiokompleksisuus on O(n) \cite{schindler2019ethdkg}. Konfiguroinnin jälkeen yksittäisen HERB osallistujan tarvitsee älysopimustoteutuksessa lähettää satunnaisluvun tuottamiseksi kaksi transaktiota \cite{cherniaeva2019homomorphic}. Itse toteutus ei ole kuitenkaan triviaali sillä osallistujilta vaaditaan kryptografista off-chain laskentaa esimerkiksi kommitointien oikeellisuuden varmentamiseksi. HERB myös vaatii lohkoketjuympäristössä osallistujilta vakuuden tai vastaavan mekanismin, sillä jos protokollaan rekisteröinti on lähes kulutonta, voi hyökkääjä osallistua protokollaan monella tilillä ja täten saada helposti yli $t$ osaa yksityisestä avaimesta. Protokollaan ei voi myöskään liittyä, eikä poistua, vapaasti. Uusien osallistujien pitää suorittaa DKG konfigurointi uudestaan. 

HERB protokollasta nostetaan esiin toisista protokollista poikkeava ominaisuus, joka on, että salaisen avaimen osien ryhmä voi olla erillinen satunnaisuutta kommitoivien ryhmästä \cite{cherniaeva2019homomorphic}. Tämä on mahdollista, sillä satunnaisuus salataan käyttäen julkista avainta. Jos kuitenkin käytännön toteutuksessa kommitointi on ilmaista, kaikille avointa ja kommitoidut salaisuudet puretaan kun niitä on saapunut esimerkiksi vähintään $t+1$, niin yksittäinen hyökkääjä voi pahimmassa tapauksessa vastata kaikista kommitoinneista ja täten päättää satunnaisen arvon. Tämä onnistuu niin, että hyökkääjä lähettää heti kommitointi-ikkunan avauduttua eri tileillä $t-1$ kommitointia, niin suurilla transaktiokuluilla, että vain hyökkääjän kommitoinnit menevät läpi. Hyökkäys olisi taloudellisesti kallis, mutta se voisi olla kannattava esimerkiksi lotto-arvonnan tapauksessa.