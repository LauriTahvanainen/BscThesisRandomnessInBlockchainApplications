\chapter{Analyysi\label{discussion}}



Taulukkoon~\ref{table:results}, on koottu käsiteltyjä metodeja, joilla lohkoketjusovellus saa hyödynnettäväkseen satunnaisuutta. Taulukossa on myös protokollan soveltamisen kannalta merkittäviä huomioita. Taulukossa on protokollan nimi, tekstimuotoisia huomioita, arvio protokollan konfiguroinnin ja yhden satunnaisluvun generoimisen kommunikaatiokompleksisuudesta, määrä joka osallistujista voi olla hyökkääjiä, niin että protokolla täyttää vaatimukset, sekä tieto siitä onko protokolla lupavapaassa lohkoketjuympäristössä Sybil resistentti. Ei Sybil resistentit protokollat vaativat käytännön toteutuksessaan mekanismin Sybil hyökkäyksen torjumiseen, jotta toteutus täyttää vaatimukset 1-4.

\begin{table}[ht]
    \caption{Metodeja satunnaisuuden saamiseen lohkoketjusovellukseen}
    \label{table:results}
    \begin{tabular}{  l  p{4cm}  p{2.5cm} p{2cm} p{4cm}}
        \toprule

\textbf{Protokolla}      
& \textbf{Huomioitavaa}
& \textbf{Kustannus}
& \textbf{Hyökkääjiä} 
& \textbf{Sybil resistentti} \\
Chainlink VRF 
& Maksaa Chainlink virtuaalivaluuttaa. Valmis älysopimus rajapinta.
& $O(1)$ Gen.
& $N$
& Kyllä  \\\hline
Lohkon tiiviste + VDF 
& VDF turvallisen viiveen määrittely. Kaksi transaktiota, lohkonumeron päättäminen ja VDF:n tuloste. Yksinkertaisin On-chain toteutus.
& $O(1)$ Konf. $O(1)$ Gen.
& $N$
& Kyllä  \\\hline
Kommitointi + VDF       
& VDF turvallisen viiveen määrittely. Vaatii kaikkien kommitoinnin.                   
&  $O(1)$ Konf. $O(1)$ Gen.
& $N-1$  
& Kyllä  \\\hline
PVSS        
& Korkea kommunikointikustannus. Monimutkainen toteutus.
& $O(N^2)$
& $N/2 - N/3$  
& Ei \\\hline
Merlin-ketju
& Kuinka toimia tuottajien kanssa, jotka eivät paljasta Merlin-ketjun arvoaan? Merlin-ketjun varmuuskopiointi.
& $O(1)$ Konf. $O(1)$ Gen. 
& $N-1$ 
& Ei, vain manipuloitavissa  \\\hline
Caucus
& Samat kuin Merlin-ketjun kanssa. Olemassaoleva älysopimustoteutus. Pienempi pysähtymisriski verrattuna Merlin-ketjuun.
& $O(N^2)$ Konf. $O(1)$ Gen. 
& $N-1$ 
& Vakuustalletus \\\hline
Homomorfinen salaus
& Vaativa off-chain laskentaan nojaava toteutus. Osittain valmis toteutus saatavilla.
& $O(N)$ Konf. $O(1)$ Gen. 
& $N/3$
& Ei  \\
        \bottomrule

    \end{tabular}
\end{table}



% Mitkä toteutettavissa PoS ketjussa?