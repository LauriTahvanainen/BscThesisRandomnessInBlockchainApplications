\chapter{Analyysi\label{discussion}}



Taulukkoon~\ref{table:results}, on koottu käsiteltyjä protokollia. Taulukossa on protokollan soveltamisen kannalta merkittäviä huomioita, arvio protokollan konfiguroinnin ja yhden satunnaisluvun generoimisen kommunikaatiokompleksisuudesta, sekä määrä joka osallistujista voi olla hyökkääjiä, niin että protokolla täyttää vaatimukset.

\begin{table}[h]
    \caption{Vaatimukset 1-4 täyttävät protokollat}
    \label{table:results}
    \begin{tabular}{  l  p{5cm}  p{4cm} p{2cm} }
        \toprule
\textbf{Protokolla}      
& \textbf{Huomioitavaa}
& \textbf{Kustannus}
& \textbf{Hyökkääjiä} \\
Lohkon tiiviste + VDF 
& VDF turvallisen viiveen määrittely. Kaksi transaktiota, lohkonumeron päättäminen ja VDF:n tuloste. Yksinkertaisin On-chain toteutus.
& $O(1)$ Konf. $O(1)$ Gen.
& $N$ \\\hline
Kommitointi + VDF       
& VDF turvallisen viiveen määrittely. Vaatii kaikkien kommitoinnin.                   
&  $O(1)$ Konf. $O(1)$ Gen.
& $N-1$  \\\hline
PVSS        
& Korkea kommunikointikustannus. Monimutkainen toteutus.
& $O(N^2)$
& $N/2 - N/3$  \\\hline
Merlin-ketju
& Kuinka toimia tuottajien kanssa, jotka eivät paljasta Merlin-ketjun arvoaan? Merlin-ketjun varmuuskopiointi.
& $O(1)$ Konf. $O(1)$ Gen. 
& $N-1$ \\\hline
Caucus
& Samat kuin Merlin-ketjun kanssa. Olemassaoleva älysopimustoteutus. Pienempi pysähtymisriski verrattuna Merlin-ketjuun.
& $O(N^2)$ Konf. $O(1)$ Gen. 
& $N-1$ \\\hline
        \bottomrule
    \end{tabular}
\end{table}

Tiivisteen ja kommitoinnin protokollat VDF:n kanssa soveltuvat erityisesti arvontatyylisiin lohkoketjusovelluksiin. Tiivisteen käyttäminen aiheuttaa yksittäiselle osallistujalle pienimmän kommunikaatiokompleksisuuden, sillä osallistujan tarvitsee vain liittyä protokollaan yhdellä transaktiolla. Kommitoinnissa osallistujan täytyy liittyä protokollaan ja kommitoida arvo, mitkä voidaan tehdä samassa transaktiossa, mutta transaktio on silti kalliimpi toteuttaa kuin tiivisteen käyttämisen tapauksessa.

VDF:n käyttämistä sovelluksissa vaikeuttaa käyttäjien ja mahdollisten hyökkääjien suuret erot laitteistojen tehokkuudessa. Viive tulisi voida valita niin, että protokolla on turvallinen tehokkaimpia laitteita vastaan, mutta kuitenkin niin, että rehellinen käyttäjä voi evaluoida funktion sovelluksen käyttötarkoituksen mukaisessa ajassa.