% \begin{abstract}{finnish}

% Tämä dokumentti on tarkoitettu Helsingin yliopiston tietojenkäsittelytieteen osaston opin\-näyt\-teiden ja harjoitustöiden ulkoasun ohjeeksi ja mallipohjaksi. Ohje soveltuu kanditutkielmiin, ohjelmistotuotantoprojekteihin, seminaareihin ja maisterintutkielmiin. Tämän ohjeen lisäksi on seurattava niitä ohjeita, jotka opastavat valitsemaan kuhunkin osioon tieteellisesti kiinnostavaa, syvällisesti pohdittua sisältöä.


% Työn aihe luokitellaan  
% ACM Computing Classification System (CCS) mukaisesti, 
% ks.\ \url{https://dl.acm.org/ccs}. 
% Käytä muutamaa termipolkua (1--3), jotka alkavat juuritermistä ja joissa polun tarkentuvat luokat erotetaan toisistaan oikealle osoittavalla nuolella.

% \end{abstract}

\begin{otherlanguage}{finnish}
\begin{abstract}
Lohkoketjusovellukset tarvitsevat satunnaisuutta. Lohkoketjun deterministisyyden ja lupavapauden takia lohkoketjusovelluksessa käytettävän satunnaisuuden täytyy olla manipuloimatonta, ennustamatonta sekä todennettavaa. Tutkielmassa käydään läpi olemassaolevia menetelmiä, joilla lohkoketjusovellus saa hyödynnettäväkseen satunnaisuutta. Menetelmät esitellään tiivisti, käydään läpi niiden oletuksia, heikkouksia sekä yksittäiselle käyttäjälle koituvia kommunikointikustannuksia. Pääpaino on menetelmissä, joissa sovelluksen käyttäjät tuottavat itse sovelluksen tarvitseman satunnaisuuden. Menetelmiä vertaillaan ja analysoidaan niiden soveltuvuutta erilaisiin lohkoketjusovelluksiin.

\end{abstract}
\end{otherlanguage}
