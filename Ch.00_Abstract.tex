% \begin{abstract}{finnish}

% Tämä dokumentti on tarkoitettu Helsingin yliopiston tietojenkäsittelytieteen osaston opin\-näyt\-teiden ja harjoitustöiden ulkoasun ohjeeksi ja mallipohjaksi. Ohje soveltuu kanditutkielmiin, ohjelmistotuotantoprojekteihin, seminaareihin ja maisterintutkielmiin. Tämän ohjeen lisäksi on seurattava niitä ohjeita, jotka opastavat valitsemaan kuhunkin osioon tieteellisesti kiinnostavaa, syvällisesti pohdittua sisältöä.


% Työn aihe luokitellaan  
% ACM Computing Classification System (CCS) mukaisesti, 
% ks.\ \url{https://dl.acm.org/ccs}. 
% Käytä muutamaa termipolkua (1--3), jotka alkavat juuritermistä ja joissa polun tarkentuvat luokat erotetaan toisistaan oikealle osoittavalla nuolella.

% \end{abstract}

\begin{otherlanguage}{finnish}
\begin{abstract}

Julkisen, todennettavan ja manipuloimattoman satunnaisuuden tuottaminen lohkoketjusovellusten tarpeisiin on haastava ongelma, johon on esitetty moninaisia ratkaisuja käyttäen niin kryptografisia menetelmiä kuin peliteoriaa. Tässä tutkielmassa esitellään erilaisia protokollia, joiden avulla lohkoketjusovelluksessa voidaan tuottaa satunnaisuutta. Yksittäisen protokollan toiminta käydään läpi avaten kuinka protokolla saavuttaa satunnaisuudelle asetetut vaatimukset, sekä arvioidaan protokollan suorittajille koituvaa kommunikointikustannusta. Lisäksi tuodaan ilmi protokollan heikkoudet, sekä huomiot, joilla on vaikutusta protokollan käyttämisessä lohkoketjusovelluksten vaihtelevissa käyttötarkoituksissa. Tutkielmaa voi hyödyntää kehittäessä lohkoketjusovellusta joka vaatii satunnaisuutta.

\end{abstract}
\end{otherlanguage}
